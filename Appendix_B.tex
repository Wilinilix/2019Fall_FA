% !TeX root = main.tex
% This is the Appendix B.

\chapter{一些习题课上讲过的题目}

\section{第1、2章习题}

	\textbf{习题1.11}\ [作业]\ \ 设$ E=\{ x=(x_n)_{n\geqslant 1} : \forall n\geqslant 1, (x_n=0)\lor(x_n=1) \}=\{ 0,1 \}^\N $. 在$ E $上定义函数
	\[
	\varphi(x)=\sum_{n\geqslant 1}\frac{2x_n}{3^{n  }}
	\]
	并在$ \{ 0,1 \} $上赋予离散拓扑(即$ d(0,1)=1 $的度量诱导的拓扑), 则在$ E $上有相应的乘积拓扑. 证明: $ \varphi $是$ E $到$ \R $的紧子集$ \varphi(E) $上的同胚.
	\begin{Proof}
	因为$ \{ 0,1 \} $是紧空间, 由Tychonoff定理可知$ E=\prod\limits_{n\geqslant 1}\{ 0,1 \} $也是紧空间, 而$ \R $是Hausdorff空间, 故只需证明$ \varphi $是连续单射即可. 在$ E $上取
	\[
	\rho(x,y)=\sup_{n\geqslant 1}\frac{d(x_n,y_n)}{n}
	\]
	那么$ \rho $是$ E $的度量且$ \rho $诱导的拓扑与乘积拓扑一致(这一结论的证明见命题\,\ref{prop:乘积拓扑空间的继承性质}\,中(3)的证明过程). 对$ 0<\delta<1 $, $ \rho(x,y)<\delta $意味着$ \forall n\geqslant 1\,(d(x_n,y_n)/n<\delta) $, 也即
	\[
	\forall n\geqslant 1\,(d(x_n,y_n)<n\delta).
	\]
	由实数的Archimedes性, 存在$ N\in\N $使得$ N\delta<1<(N+1)\delta $, 从而只需$ x_n=y_n $对$ n=1,2,\cdots,N $成立即可, 也即
	\[
	B_\rho(x,\delta)\subset\{ y : x_n=y_n, 1\leqslant n\leqslant N \}.
	\]
	那么$ \forall\varepsilon>0 $, 存在$ n_0\in\N $使得$ \sum\limits_{n\geqslant n_0+1}\frac{2}{3^n}<\varepsilon $, 而注意到
	\[
	\diam(\varphi(B_\rho(x,\delta)))\leqslant\diam(\varphi\{ y : x_n=y_n, 1\leqslant n\leqslant N \})<\sum_{n\geqslant N+1}\frac{2}{3^n},
	\]
	取足够小的$ \delta $使得$ N\geqslant n_0 $, 那么此时上式右侧$ <\varepsilon $, 于是$ \varphi $连续.
	
	而$ \forall x,y\in E $且$ x\ne y $, 取最小的使得$ x_k\ne y_k $的正数$ k $, 那么
	\[
	\abs{\varphi(y)-\varphi(x)}\geqslant\frac{2\abs{x_k-y_k}}{3^k}-\sum_{n\geqslant k+1}\frac{2}{3^n}=\frac{2}{3^k}-\frac{1}{3^k}=\frac{1}{3^k}>0,
	\]
	即$ \varphi(x)\ne\varphi(y) $, 故$ \varphi $是单射.\qed
	\end{Proof}

	\textbf{习题2.2}\ [习题课]\ \ 证明度量空间$ (E,d) $是完备的充分必要条件是: 对$ E $中任一序列$ (x_n)_{n\geqslant 1} $, 若对$ \forall n\geqslant 1 $, 有$ d(x_n,x_{n+1})<2^{-n} $, 则序列$ (x_n)_{n\geqslant 1} $收敛.
	\begin{Proof}
	\textsl{必要性}. 由题设可知对$ \forall n,p\geqslant 1 $, 有
	\[
	d(x_n,x_{n+p})\leqslant\sum_{k=n}^{n+p-1}d(x_k,x_{k+1})<\sum_{k=n}^{n+p-1}2^{-k}<\sum_{k=n}^\infty 2^{-k}=2^{1-n}
	\]
	可知$ n, p\to\infty $时上式趋于0, 从而$ (x_n)_{n\geqslant 1} $是Cauchy列. 由度量空间完备可知$ (x_n)_{n\geqslant 1} $收敛.
	
	\textsl{充分性}. 设$ (y_n)_{n\geqslant 1} $是$ (E,d) $上的Cauchy列, 则可取子列$ (y_{n_k})_{k\geqslant 1} $使得$ \forall k\geqslant 1 $有$ d(y_{n_k},y_{n_{k+1}})<2^{-k} $. 由题设可知$ (y_{n_k})_{k\geqslant 1} $收敛, 那么$ (y_n)_{n\geqslant 1} $也收敛.\qed
	\end{Proof}
	
	\textbf{习题2.3}\ [作业]\ \ 设$ (E,d) $是度量空间, $ (x_n)_{n\geqslant 1} $是$ E $中的Cauchy列, 并有$ A\subset E $. 假设$ A $的闭包$ \bar{A} $在$ E $中完备且有$ \lim\limits_{n\to\infty}d(x_n,A)=0 $. 证明: $ (x_n)_{n\geqslant 1} $在$ E $中收敛.
	\begin{Proof}
	由$ d(x,A) $的定义与$ \lim\limits_{n\to\infty}d(x_n,A)=0 $可知
	\[
	\forall n\geqslant 1\,\exists y_n\in A\,\left(d(x_n,y_n)<d(x_n,A)+\frac{1}{n}\right).
	\]
	则$ \forall n,m\geqslant 1 $, 有
	\begin{align*}
	d(y_n,y_m)&\leqslant d(y_n,x_n)+d(x_n,x_m)+d(x_m,y_m)\\
	&\leqslant d(x_n,A)+\frac{1}{n}+d(x_n,x_m)+d(x_m,A)+\frac{1}{m}\to 0\qquad (n,m\to\infty)
	\end{align*}
	于是$ (y_n)_{n\geqslant 1}\subset A\subset\bar{A} $是Cauchy列. 由$ \bar{A} $完备可知$ (y_n)_{n\geqslant 1} $收敛于某个$ y_0\in\bar{A} $. 从而
	\[
	d(x_n,y_0)\leqslant d(x_n,y_n)+d(y_n,y_0)\leqslant\frac{1}{n}+d(x_n,A)+d(y_n,y_0)\to 0\qquad(n\to\infty)
	\]
	于是$ (x_n)_{n\geqslant 1} $在$ E $中收敛.\qed
	\end{Proof}
	
	\textbf{习题2.4}\ [作业]\ \ 设$ (E,d) $是度量空间, $ \alpha>0 $. 设$ A\subset E $满足$ \forall x,y\in A $且$ x\ne y $必有$ d(x,y)\geqslant\alpha $. 证明: $ A $是完备的.
	\begin{Proof}
	设$ (x_n)_{n\geqslant 1}\subset A $是Cauchy列, 则
	\[
	\forall 0<\varepsilon<\alpha\,\exists n_0\in\N\,(n,m\geqslant n_0\Rightarrow d(x_n,x_m)<\varepsilon<\alpha),
	\]
	从而由题设可知只能$ x_n=x_m $. 即$ \forall n\geqslant n_0\,(x_n=x_{n_0}) $. 从而$ (x_n)_{n\geqslant 1} $收敛且极限为$ x_{n_0} $, 故$ A $完备.\qed
	\end{Proof}
	
	\textbf{习题2.6}\ [习题课]\ \ 设$ (E,d) $是度量空间, $ (x_n)_{n\geqslant 1} $是$ E $中发散的Cauchy列. 证明:
	
	(1) 任取$ x\in E $, 序列$ (d(x,x_n))_{n\geqslant 1} $收敛到一个正数, 记作$ g(x) $;
	
	(2) 函数$ x\mapsto1/g(x) $是一个从$ E $到$ \R $的连续函数;
	
	(3) 上面定义的函数无界.
	\begin{Proof}
	(1) $ \forall n,m\geqslant 1 $, 有
	\[
	\abs{d(x,x_n)-d(x,x_m)}\leqslant d(x_n,x_m)\to 0\qquad (n,m\to\infty)
	\]
	从而$ (d(x,x_n))_{n\geqslant 1} $是$ \R $上的Cauchy列. 于是它收敛, 并设其极限是$ \lambda $, 则有$ \lambda\geqslant 0 $. 若$ \lambda=0 $, 则有$ x_n\to x $, 这与$ (x_n)_{n\geqslant 1} $发散矛盾, 从而只能$ \lambda>0 $.
	
	(2) 注意到函数$ t\mapsto 1/t $是连续的, 只需证明$ g $是连续函数. 对$ \forall x,y\in E $, 由
	\[
	\abs{g(x)-g(y)}=\abs{\lim\limits_{n\to\infty}d(x,x_n)-\lim\limits_{n\to\infty}d(y,x_n)}=\lim\limits_{n\to\infty}\abs{d(x,x_n)-d(y,x_n)}\leqslant d(x,y)
	\]
	可知$ g $是一致连续的, 从而$ g $连续.
	
	(3) 因为$ (x_n)_{n\geqslant 1} $是Cauchy列, 则
	\[
	\forall\varepsilon>0\,\exists n_0\in\N\,(n\geqslant n_0\Rightarrow d(x_m,x_{n_0})<\varepsilon)
	\]
	则$ \lim\limits_{n\to\infty}d(x_n,x_{n_0})\leqslant\varepsilon $. 从而$ g(x_{n_0})\leqslant\varepsilon $, 于是$ 1/g(x_{n_0})\geqslant 1/\varepsilon $. 由$ \varepsilon $的任意性可知$ x\mapsto 1/g(x) $无界.\qed
	\end{Proof}
	
	\textbf{习题2.8}\ [作业]\ \ 设 $ f: \R^{n}\to\R $ 是一致连续函数, 证明存在两个非负常数 $ a $ 和 $ b $, 使得
	\[
		\abs{f(x)}\leqslant a\norm{x}+b.
	\]
	这里 $ \norm{x} $ 是 $ x $ 的 Euclid 范数.

	\begin{Proof}
		由于 $ f $ 是一致连续的, 取 $ \varepsilon=1 $, 则
		\[
			\exists \delta>0\,(\norm{x-y}<\delta\Rightarrow\abs{f(x)-f(y)}<1),
		\]
		由实数的 Archimedes 性知 $ \exists n\in\N\,(n\delta\leqslant\norm{x}<(n+1)\delta) $, 则此时取 $ 0 $ 到 $ x $ 连线上划分
		\[
			x_{0}=0,\quad x_{1}=\delta\cdot\frac{x}{\norm{x}},\quad x_{2}=2\delta\cdot\frac{x}{\norm{x}}, \dots, x_{n}=n\delta\cdot\frac{x}{\norm{x}},\quad x_{n+1}=x.
		\]
		则 
		\[
			\begin{aligned}
				\abs{f(x)-f(0)} & \leqslant\sum_{k=0}^{n}\abs{f(x_{k+1})-f(x_{k})}\\
				& \leqslant n+1\\
				& = n\cdot\frac{1}{\delta}\cdot\delta+1\\
				& \leqslant\frac{1}{\delta}\cdot\norm{x}+1.
			\end{aligned}
		\]
		故 $ \abs{f(x)}\leqslant\abs{f(0)}+\norm{x}/\delta+1 $. 取 $ a=1/\delta, b=\abs{f(0)}+1 $ 即可.\qed
	\end{Proof}
	
	\begin{Remark}
		本题结论在1维情形下, 即是一致连续函数在无穷远处是线性增长的, 这是一个较强的结论, 它说明一个一致连续函数在充分远处可以使用线性函数控制.
	\end{Remark}
	
	\textbf{习题2.10}\ [作业]\ \ 构造一个反例说明, 在压缩映照原理\,\ref{thm:压缩映照原理}\,中, 如果我们把映射 $ f $ 满足的条件减弱为
	\[
		d(f(x), f(y))<d(x, y)\qquad \forall x, y\in E\wedge x\neq y,
	\]
	则结论不成立.

	\begin{Solution}
		取 $ E=\R $, $ f(x)=\sqrt{x^{2}+1} $, 则
		\[
			\begin{aligned}
				d(f(x), f(y)) & =\abs{\sqrt{x^{2}+1}-\sqrt{y^{2}+1}}\\
				& \leqslant\abs{\frac{\xi}{\sqrt{\xi^{2}+1}}}\cdot\abs{x-y}<\abs{x-y}.
			\end{aligned}
		\]
		但 $ f(x)=x $ 显然无解.
	\end{Solution}
	\begin{Remark}
		本题也可以举出另外的反例. 考虑无解的方程 $ \arctan x=\pi/2 $, 那么可以构造映射
		\[
			f : \R\to\R,\qquad x\mapsto x+\frac\pi 2-\arctan x,
		\]
		注意到
		\[
			f'=1-\frac{1}{\xi^2+1}=\frac{\xi^2}{\xi^2+1}<1,
		\]
		可知如此构造的 $ f $ 满足题设条件, 但方程 $ f(x)=x $ 等价于 $ \arctan x=\pi/2 $ 无解, 从而不存在不动点.
	\end{Remark}
	
	\textbf{习题2.10$\bm{'}$}\ [习题课]\ \ 设$ (E,d) $是紧的度量空间, 在压缩映照原理中若将映射$ f $满足的条件减弱到
	\[
	\forall x,y\in E\,,x\ne y\,(d(f(x),f(y))<d(x,y))
	\]
	则$ f $仍然存在唯一不动点.
	\begin{Proof}
	记$ g(x)=d(x,f(x)) $, 由$ f $连续且$ d $连续可知$ g $也是连续的. 因为$ E $是紧的, 故$ g(E) $也是紧的, 从而$ g(E) $能取到最小值$ \lambda $. 反设$ \lambda>0 $, 则$ \exists x_0\in E\,(d(x_0,f(x_0))=\lambda) $, 那么
	\[
	d(f(x_0),f^2(x_0))<d(x_0,f(x_0))=\lambda.
	\]
	这与$ \lambda $是$ g $的最小值矛盾. 从而只能$ \lambda=0 $, 此时$ x_0 $即为$ f $的不动点.
	
	再说明唯一性. 若$ x_0\ne y_0 $都是$ f $的不动点, 由
	\[
	d(f(x_0),f(y_0))=d(x_0,y_0)
	\]
	矛盾.\qed
	\end{Proof}
	
	\textbf{习题2.11}\ [习题课]\ \ 设$ (E,d) $是一个完备的度量空间, $ f $是其上的映射, 且满足$ f^n $是压缩映射(这里$ f^n $表示$ f $的$ n $次复合). 证明: $ f $有唯一的不动点, 并给出例子说明$ f $可以不连续.
	\begin{Proof}
	因$ f^n $是压缩映射, 由压缩映照原理可知$ f^n $有唯一的不动点$ x_0 $, 也即$ f^n(x_0)=x_0 $, 那么
	\[
	f^n(f(x_0))=f^{n+1}(x_0)=f(f^n(x_0))=f(x_0),
	\]
	从而$ f(x_0) $也是$ f^n $的不动点. 由$ f^n $不动点的唯一性可知只能$ x_0=f(x_0) $, 即$ x_0 $是$ f $的不动点.
	
	再说明唯一性. 若$ x_0\ne y_0 $都是$ f $的不动点, 则$ f^n(y_0)=y_0 $, 即$ y_0 $也是$ f^n $的不动点, 从而$ x_0=y_0 $.
	
	取$ f=1_\Q $, 那么注意到$ f^2\equiv 1 $是压缩映射, 但$ f $并不连续.\qed
	\end{Proof}
	
	\textbf{习题2.12}\ [习题课]\ \ 记区间$ I=(0,\infty) $上的自然拓扑为$ \tau $.
	
	(1) 证明$ \tau $可以被以下完备的距离$ d $诱导:
	\[
	d(x,y)=\abs{\log x-\log y};
	\]
	
	(2) 设函数$ f : I\to I $一次连续可微, 且满足对某个$ \lambda<1 $, 任取$ x\in I $都有$ x\abs{f'(x)}\leqslant\lambda f(x) $. 证明$ f $在$ I $上存在唯一的不动点.
	\begin{Proof}
	(1) $ d $是距离是显然的, 下证它完备: 任取$ (I,d) $中的Cauchy列$ (x_n)_{n\geqslant 1} $, 那么
	\[
	\forall\varepsilon>0\,\exists n_0\in\N\,(n,m\geqslant n_0\Rightarrow d(x_n,x_m)=\abs{\log{x_n}-\log{x_m}}<\varepsilon)
	\]
	则$ (\log x_n)_{n\geqslant 1} $是$ \R $中的Cauchy列, 从而存在$ z\in\R $使得
	\[
	\lim_{n\to\infty}\abs{\log x_n-z}=0,
	\]
	也即$ \lim\limits_{n\to\infty}\abs{x_n-\exp z}=0 $, 从而$ (x_n)_{n\geqslant 1} $收敛到$ \exp z $, 于是$ (I,d) $完备.
	
	再说明$ \tau $可以被$ d $诱导. 记$ d $诱导的拓扑为$ \tau_d $, 则$ \forall r>0,x\in I $, 考虑$ (I,d) $中的球$ B_0(x,r)=\{ y : d(y,x)<r \} $. 而
	\[
	d(y,x)<r\Longleftrightarrow\abs{\log y-\log x}<r\Longleftrightarrow y\in(x\exp (-r),x\exp r),
	\]
	于是$ \tau_d\subset \tau $. 而对$ I $中任意开区间$ (a,b) $, 设$ x=\sqrt{ab} $且$ r=\frac{1}{2}\log\frac{b}{a} $, 那么$ (a,b)=(x\exp(-r),x\exp r) $. 于是$ \tau\subset\tau_d $. 这说明$ \tau=\tau_d $.
	
	(2) 由Cauchy中值定理可知$ \forall x,y\in I $, 不妨$ x<y $, 存在$ \xi\in(x,y) $使得
	\[
	\abs{\frac{\log f(x)-\log f(y)}{\log x-\log y}}=\abs{\frac{f'(\xi)/f(\xi)}{1/\xi}}\leqslant\lambda.
	\]
	故$ f $是压缩映射, 故存在唯一不动点.\qed
	\end{Proof}
	\begin{Remark}
	(2) 有另证: 由题设可知
	\[
	\pm\frac{f'(t)}{f(t)}\leqslant\frac{\lambda}{t},\qquad t\in I,
	\]
	两侧同时在$ [x,y] $上积分得
	\[
	\pm(\log f(y)-\log f(x))\leqslant\lambda(\log y-\log x),
	\]
	从而$ \abs{\log f(y)-\log f(x)}\leqslant\lambda\abs{\log y-\log x} $. 因此$ f : I\to I $是完备度量空间$ (I,d) $上的压缩映射. 因此由压缩映照原理可知$ f $在$ I $上存在唯一不动点.
	
	并且注意到$ I $上的Euclid距离是不完备的, 尽管$ d $与Euclid距离诱导出的拓扑是相同的, 但$ d $却是完备的. 这说明完备性并不是一个拓扑概念, 它跟空间上赋予的度量有关.
	\end{Remark}
	
	\textbf{习题2.15}\ [习题课]\ \ 设$ (E,d) $是完备度量空间, $ f $和$ g $是$ E $上两个可交换的压缩映射(即$ fg=gf $). 证明$ f $和$ g $有唯一的共同不动点. 并举出反例说明当可交换条件不满足时结论不成立.
	\begin{Proof}
	设$ f $的不动点是$ x_0 $, 即$ x_0=f(x_0) $. 那么
	\[
	g(x_0)=g(f(x_0))=f(g(x_0)),
	\]
	即$ g(x_0) $也是$ f $的不动点, 从而$ g(x_0)=x_0 $, $ x_0 $是$ g $的不动点. 由对称性可证另一侧.
	
	若去掉可交换的条件, 取$ f\equiv\frac{1}{4} $而$ g\equiv\frac{3}{4} $即可. 此时注意到
	\[
	f(g(x))=\frac{1}{4},\qquad g(f(x))=\frac{3}{4},
	\]
	也即$ f $与$ g $的不动点分别是$ \frac{1}{4} $与$ \frac{3}{4} $.\qed
	\end{Proof}
	
\section{第3章习题}

	\textbf{习题3.2}\ [习题课]\ \ 设 $ E $ 是 $ \R $ 上所有的实系数多项式构成的线性空间, 对任一 $ p\in E $, 定义
	\[
		\norm{p}_{\infty}=\max_{x\in[0, 1]}\abs{p(x)}.
	\]
	\begin{enumerate}[(1)]
		\item 证明 $ \norm{\cdot}_{\infty} $ 是 $ E $ 上的范数.
		\item 任取一个 $ a\in\R $, 定义线性映射 $ L_{a}:E\to \R $ 满足 $ L_{a}(p)=p(a) $. 证明 $ L_{a} $ 连续的充分必要条件是 $ a\in[0, 1] $, 并且给出该连续线性映射的范数.
		\item 设 $ a<b $ 并定义 $ L_{a, b}:E\to \R $ 满足
		\[
			L_{a, b}(p)=\int_{a}^{b}p(x)\diff x,
		\]
		给出 $ a, b $ 的取值范围, 使其成为 $ L_{a, b} $ 连续的充分必要条件, 然后确定 $ L_{a, b} $ 的范数.
	\end{enumerate}

	\begin{Proof}
		(1) 验证范数的4条性质:
		\begin{itemize}
			\item $ \norm{p}_{\infty}\geqslant 0 $ 显然成立;
			\item $ \norm{p}_{\infty}=0 \Rightarrow \forall x\in [0, 1]\,(\abs{p(x)}=0) $, 由代数基本定理可知 $ \forall x\in\R\,(p(x)=0) $, 即 $ p=0 $;
			\item $ \norm{\lambda p}_{\infty}=\abs{\lambda}\norm{p}_{\infty} $ 显然成立;
			\item $ \norm{p+q}_{\infty}\leqslant\norm{p}_{\infty}+\norm{q}_{\infty} $ 显然成立.
		\end{itemize}

		(2) \textsl{必要性}. 若 $ L_{a} $ 是连续的, 则
		\[
			\abs{L_{a}(p)}=\abs{p(a)}\leqslant \norm{L_{a}}\norm{p}_{\infty}\qquad\norm{L_{a}}<\infty.
		\]
		取 $ p_{n}=x^{n} $, 则有
		\[
			\norm{L_{a}(p_{n})}=\abs{a^{n}}\leqslant\abs{a}^{n}\leqslant \norm{L_{a}}\norm{x^{n}}_{\infty},
		\]
		而由 $ \norm{p_{n}}_{\infty}=1 $, 知
		\[
			\abs{a}^{n}\leqslant\norm{L_{a}},
		\]
		令 $ n\to\infty $, 则有 $ \abs{a}\leqslant1 $. 

		同理取 $ p_{n}=(1-x)^{n} $ , 类似可得 $ \abs{1-a}\leqslant1 $, 从而 $ a\in[0, 1] $.

		\textsl{充分性}. 由 $ a\in[0, 1] $ 可知
		\[
			\forall p\in E\,\big(\abs{L_{a}(p)}=\abs{p(a)}\leqslant\norm{p}_{\infty}\big).
		\]
		则 $ L_{a} $ 连续.

		下面计算 $ \norm{L_{a}} $. 由充分性证明的过程与注\,\ref{rmk:范数性质}\,的\,\ref{rmk:范数性质最小C}\,可知 $ \norm{L_{a}}\leqslant1 $, 取 $ p(x)\equiv 1 $, 则 $ \norm{p}_{\infty}=1 $, 且 $ \abs{L_{a}(p)}=1 $, 故 $ \norm{L_{a}}=1 $.

		(3) $ a, b\in[0, 1] $, 且 $ \norm{L_{a, b}}=b-a $. 下面给出证明.

		\textsl{必要性}. 若 $ L_{a, b} $ 连续, 则
		\[
			\forall p\in E\,\bigg(\abs{L_{a, b}(p)}=\abs{\int_{a}^{b}p(x)\diff x}\leqslant \norm{L_{a, b}}\norm{p}_{\infty}\bigg)\qquad \norm{L_{a, b}}<\infty.
		\]
		取 $ p_{n}=x^{n} $, $ \norm{p_{n}}_{\infty}=1 $, 则
		\[
			\abs{\int_{a}^{b}x^{n}\diff x}=\abs{\frac{b^{n+1}-a^{n+1}}{n+1}}\leqslant\norm{L_{a, b}}<\infty.
		\]
		若 $ \abs{a}<\abs{b} $, 则
		\[
			\frac{\abs{b}^{n+1}}{n+1}\left( 1-\abs{\frac{a}{b}}^{n+1} \right) \leqslant \norm{L_{a, b}}<\infty,
		\]
		令 $ n\to\infty $ 知 $ \abs{b}\leqslant1 $, 因此 $ \abs{a}\leqslant1, \abs{b}\leqslant1 $. 同理 $ \abs{b}<\abs{a} $ 时, 亦有 $ \abs{a}\leqslant1, \abs{b}\leqslant1 $; 当 $ \abs{a}=\abs{b} $ 时, 因为 $ a<b $, 所以 $ a=-b $, 也可以得出 $ \abs{a}\leqslant1, \abs{b}\leqslant1 $

		与 (2) 一样, 再取 $ p_{n}=(1-x)^{n} $, 可以得到 $ \abs{a-1}\leqslant1, \abs{b-1}\leqslant1 $.

		\textsl{充分性}. 若 $ a, b\in[0, 1] $, 则有
		\[
			\abs{L_{a, b}(p)}=\abs{\int_{a}^{b}p(x)\diff x}\leqslant \abs{b-a}\cdot \norm{p}_{\infty},
		\]
		则 $ L_{a, b} $ 连续. 

		因为 $ \norm{L_{a, b}}<b-a $, 而当 $ p(x)\equiv 1 $ 时 $ \abs{L_{a, b}(p)}=b-a $, 故 $ \norm{L_{a, b}}=b-a $.\qed
	\end{Proof}

	\textbf{习题3.3}\ [作业]\ \ 设$ (E,\norm{\cdot}_\infty) $是习题3.2中定义的赋范空间, 设$ E_0 $是$ E $中常数项为0的多项式构成的线性子空间(即$ p\in E_0\Longleftrightarrow p(0)=0 $).
	\begin{enumerate}[(1)]
	\item 证明$ N(p)=\norm{p'}_\infty $定义了$ E_0 $上的一个范数, 并且对任意$ p\in E_0 $, 有$ \norm{p}_\infty\leqslant N(p) $;
	\item 证明$ L(p)=\int_0^1\frac{p(x)}{x}\diff x $定义了$ E_0 $上关于$ N $的连续线性泛函, 并求出它的范数;
	\item 上面定义的$ L $是否关于$ \norm{\cdot}_\infty $连续?
	\item 范数$ \norm{\cdot}_\infty $和$ N $在$ E_0 $上是否等价?
	\end{enumerate}
	\begin{Proof}
	(1) 先证明$ N(p) $是$ E_0 $上的范数, 注意到
	\[
	N(p)=0\Longleftrightarrow p'=0\Longleftrightarrow p=c,
	\]
	其中$ c $是一个常数, 由$ p(0)=0 $可知$ c=0 $, 从而$ p=0 $, 正定性成立. 再由
	\[
	N(\lambda p)=\norm{\lambda p'}_\infty=\abs{\lambda}\cdot\norm{p'}_\infty=\abs{\lambda}N(p)
	\]
	与
	\[
	N(p+q)=\norm{p'+q'}_\infty\leqslant\norm{p'}_\infty+\norm{q'}_\infty\leqslant N(p)+N(q)
	\]
	可知齐次性与三角不等式也成立, 从而$ N(p) $是$ E_0 $上的范数.
	
	再设$ p\in E_0 $, 任取$ x\in[0,1] $, 由Lagrange中值定理可知
	\[
	\abs{\frac{p(x)-p(0)}{x-0}}=\abs{p'(\xi)}\leqslant\norm{p'}_\infty=N(p),
	\]
	也即$ \abs{p(x)}\leqslant \norm{p'}_\infty\abs{x}\leqslant \norm{p'}_\infty $这说明$ \norm{p}_\infty\leqslant\norm{p'}_\infty=N(p) $.
	
	(2) 由积分算子的线性性可知$ L(p) $是线性泛函, 且由Lagrange中值定理可知
	\[
	\abs{\int_0^1\frac{p(x)}{x}\diff x}\leqslant\sup_{0\leqslant x\leqslant 1}\abs{\frac{p(x)}{x}}\leqslant\sup_{0\leqslant x\leqslant 1}\abs{p'(x)}=N(p),
	\]
	从而$ L(p)\leqslant N(p) $, 这说明$ L $关于范数$ N $是连续的, 且有$ \norm{L}\leqslant 1 $. 再取$ p(x)=x\in E_0 $, 此时$ L(p)=1=N(p) $, 从而$ \norm{L}=1 $
	
	(3) $ L $关于$ \norm{\cdot}_\infty $不连续. 用反证法证明, 假设$ L $关于$ \norm{\cdot}_\infty $连续, 设
	\[
	 F_0=\{ f\in C[0,1] : f(0)=0 \},
	\]
	容易验证$ F_0 $是一个Banach空间. 由Weierstrass逼近定理可知$ \forall f\in C[0,1] $, 可以使用多项式一致逼近, 从而$ E_0 $在$ F_0 $中稠密, 于是$ E_0 $上的连续线性泛函可以唯一地扩张成$ F_0 $上的连续线性泛函, 将扩张后的连续线性泛函记作$ \tilde{L} $, 那么取$ F_0 $中的元素$ f(x)=x^{1/n} $后注意到$ \norm{f}_\infty=1 $且$ \tilde{L}(f)=n $可知$ \tilde{L} $关于$ \norm{\cdot}_\infty $不连续, 矛盾.
	
	(4) 若两范数等价, 那么它们诱导出的拓扑应当是相同的, 而(3)已经说明了它们诱导的拓扑不相同, 于是两范数不等价.\qed
	\end{Proof}
	
	\textbf{习题3.4}\ [习题课]\ \ 设$ E $是由$ [0,1] $上所有连续函数构成的线性空间, 定义$ E $的两个范数分别为$ \norm{f}_1=\int_0^1\abs{f(x)}\diff x $和$ N(f)=\int_0^1x\abs{f(x)}\diff x $.
	\begin{enumerate}[(1)]
	\item 验证$ N $的确是$ E $上的范数, 且$ N\leqslant\norm{\cdot}_1 $.
	\item 设函数
	\[
	f_n(x)=\begin{cases}
	n-n^2x & ,x\leqslant 1/n\\
	0 & ,\text{其他}
	\end{cases}
	\]
	证明函数列$ (f_n)_{n\geqslant 1} $在$ (E,N) $中收敛到0, 它在$ (E,\norm{\cdot}_1) $中是否收敛? 由这两个范数在$ E $上诱导的拓扑是否相同?
	\item 设$ a\in(0,1] $, 并令$ B=\{ f\in E : f(x)=0, \forall x\in[0,a] \} $. 证明这两个范数在$ B $上诱导相同的拓扑.
	\end{enumerate}
	
	\begin{Proof}
	(1) 首先我们需要说明$ N $确实是一个范数. 其中正定性由
	\[
	N(f)=0\Longleftrightarrow xf(x)=0\Longleftrightarrow f=0
	\]
	可知, 而齐次性与三角不等式是显然的, 从而$ N $是$ E $上的一个范数. 所求证不等式由
	\[
	N(f)=\int_0^1x\abs{f(x)}\diff x\leqslant\int_0^1\abs{f(x)}\diff x=\norm{f}_1
	\]
	对任意$ f\in C[0,1] $成立可知.
	
	(2) 注意到
	\[
	N(f_n)=\int_0^{1/n}x(n-n^2x)\diff x=\int_0^{1/n}nx(1-nx)\diff x=\frac{1}{n}\int_0^1t(1-t)\diff t=\frac{1}{6n},
	\]
	从而由$ \lim\limits_{n\to\infty}N(f_n)=0 $可知$ (f_n)_{n\geqslant 1} $依范数$ N $收敛到0. 若 $ f_{n} $ 收敛, 不妨设$ (f_n)_{n\geqslant 1} $依范数$ \norm{\cdot}_1 $收敛到$ f\in E $, 那么
	\[
	\norm{f-f_n}_1=\int_0^{1/n}\abs{n(1-nx)-f(x)}\diff x+\int_{1/n}^1\abs{f(x)}\diff x\geqslant\int_{1/n}^1\abs{f(x)}\diff x,
	\]
	上式中令$ n\to\infty $可得
	\[
	\int_0^1\abs{f(x)}\diff x=0,
	\]
	也即$ f=0 $, 但此时
	\[
	\norm{f_n-f}_1=\norm{f_n}_{1}=\int_0^{1/n}\abs{n(1-nx)}\diff x=\frac{1}{2},
	\]
	这与它依范数$ \norm{\cdot}_1 $收敛到$ f $矛盾. 从而$ (f_n)_{n\geqslant 1} $不依范数$ \norm{\cdot}_1 $收敛. 这也说明了这两个范数在$ E $上诱导的拓扑不同.
	
	(3) 由于等价的范数诱导相同的拓扑, 只需证明$ N $与$ \norm{\cdot}_1 $在空间$ B $上等价即可. 由(1)已知$ N\leqslant\norm\cdot_1 $成立, 而
	\[
	N(f)=\int_0^1x\abs{f(x)}\diff x=\int_a^1x\abs{f(x)}\diff x\geqslant a\int_a^1\abs{f(x)}\diff x\geqslant a\int_0^1\abs{f(x)}\diff x=a\norm{f}_1.
	\]
	也即$ \norm{\cdot}_1\leqslant\frac{1}{a}N $, 从而$ \norm{\cdot}_1 $与$ N $在$ B $上是等价的.\qed
	\end{Proof}

	\textbf{习题3.5}\ [习题课]\ \ 设 $ \varphi:[0, 1]\to [0, 1] $ 连续函数并且不恒等于 1. 设 $ \alpha\in\R $, 定义 $ C([0, 1],\R) $ \footnote{这里 $ C([0, 1],\R) $ 表示从 $ [0, 1] $ 到 \R 上的连续函数的全体 }上的映射 $ T $ 为
	\[
		T(f)(x)=\alpha+\int_{0}^{x}f(\varphi(t))\diff t.
	\]
	证明 $ T^2 $ 是压缩映射. 再根据以上结论证明下面的方程存在唯一解:
	\begin{equation}\label{eq:3.5题公式}
		f(0)=\alpha, f'(x)=f(\varphi(x))\qquad x\in[0, 1].
	\end{equation}

	\begin{Proof}
		对$\forall f, g\in C[0, 1]$, $ \forall x\in[0, 1] $ 都有
		\[
			\begin{aligned}
				\norm{Tf-Tg} & =\sup_{0\leqslant x\leqslant 1}\int_{0}^{x}(f(\varphi(t))-g(\varphi(t)))\diff t \leqslant x\cdot \sup_{0\leqslant x\leqslant 1}\abs{f(x)-g(x)} = x\cdot \norm{f-g}.
			\end{aligned}
		\]
		进一步 
		\[
			\begin{aligned}
				\norm{T^{2}f-T^{2}g} & =\sup_{0\leqslant x\leqslant 1}\abs{\int_{0}^{x}(Tf)(\varphi(t))-(Tg)(\varphi(t))\diff t} \\
				& \leqslant\sup_{0\leqslant x\leqslant 1}\int_{0}^{x}\varphi (t)\norm{f-g}\diff t  =\norm{f-g}\int_{0}^{x}\varphi(t)\diff t  \leqslant\norm{f-g}\int_{0}^{1}\varphi(t)\diff t \leqslant\lambda\cdot\norm{f-g},
			\end{aligned}
		\]
		其中 $ \lambda=\int_{0}^{1}\varphi(t)\diff t<1 $, 即 $ T^{2} $ 是压缩映射, 则 $ T $ 有唯一不动点.

		而方程\,\eqref{eq:3.5题公式}\,成立 $ \Longleftrightarrow $ $ f(x)=\alpha +\int_{0}^{x} f(\varphi(t))\diff t $. 即 $ Tf=f $, 故方程\,\eqref{eq:3.5题公式}\,有唯一解. 
	\end{Proof}
	
	\textbf{习题3.6}\ [习题课]\ \ 设$ \alpha\in\R,\ a>0,\ b>1 $, 考察如下微分方程
	\begin{equation}\label{eq:3.6题公式}
	f(0)=\alpha,\qquad f'(x)=af(x^b),\qquad x\in[0,1]
	\end{equation}
	\begin{enumerate}[(1)]
	\item 令$ M>0 $, 验证$ E=C([0,1],\R) $上赋予范数
	\[
	\norm{f}=\sup_{0\leqslant x\leqslant 1}\abs{f(x)}\exp(-Mx)
	\]
	后成为一个Banach空间.
	\item 定义映射
	\[
	T : E\to E,\qquad f(x)\mapsto\alpha+\int_0^x af(t^b)\diff t,
	\]
	证明: 选取合适的$ M $后可以使得$ T $是压缩映射.
	\item 证明微分方程\,\eqref{eq:3.6题公式}\,有唯一解.
	\end{enumerate}
	\begin{Proof}
	(1) 易证$ \norm{\cdot} $是$ E $上的范数, 并且注意到$ (E,\norm{\cdot}_\infty) $是Banach空间. 由
	\[
	\norm{f}=\sup_{0\leqslant x\leqslant 1}\abs{f(x)}\exp(-Mx)\leqslant\norm{f}_\infty
	\]
	与
	\[
	\norm{f}=\sup_{0\leqslant x\leqslant 1}\abs{f(x)}\exp(-Mx)\geqslant\exp(-M)\norm{f}_\infty
	\]
	可知$ \exp(-M)\norm{\cdot}_\infty\leqslant\norm{\cdot}\leqslant\norm{\cdot}_\infty $, 从而$ \norm{\cdot} $与$ \norm{\cdot}_\infty $等价, 这说明$ (E,\norm{\cdot}) $也是一个Banach空间.
	
	(2) 对任意的$ f, g\in E $与$ x\in[0,1] $, 有
	\[
	\begin{aligned}
	\norm{Tf-Tg} & =\sup_{0\leqslant x\leqslant 1}\abs{\int_0^xa(f(t^b)-g(t^b))\diff t}\cdot\exp(-Mx) \\
	& \leqslant a\cdot\sup_{0\leqslant x\leqslant 1}\int_0^x\abs{f(t^b)-g(t^b)}\exp(-Mt)\exp(Mt)\diff t\cdot\exp(-Mx) \\
	& \leqslant a\cdot\sup_{0\leqslant x\leqslant 1}\int_0^x\abs{f(t^b)-g(t^b)}\exp(-Mt)\diff t\cdot\exp(Mx)\exp(-Mx) \\
	& =a\cdot\sup_{0\leqslant x\leqslant 1}\int_0^x\abs{f(t^b)-g(t^b)}\exp(-Mt)\diff t\\
	& =a\cdot\sup_{0\leqslant x\leqslant 1}\int_0^{x^b}\abs{f(u)-g(u)}\exp(-Mu^{1/b})\frac{u^{1/b-1}}{b}\diff u\\
	& =a\cdot\sup_{0\leqslant x\leqslant 1}\int_0^{x^b}\abs{f(u)-g(u)}\exp(-Mu)\exp(-M(u^{1/b}-u))\frac{u^{1/b-1}}{b}\diff u\\
	& =a\cdot\sup_{0\leqslant x\leqslant 1}\abs{f(x^b)-g(x^b)}\exp(-Mx^b)\int_0^x\exp(-M(t-t^b))\diff t\\
	& \leqslant a\cdot\norm{f-g}\int_0^1\exp(-M(t-t^b))\diff t.
	\end{aligned}
	\]
	因为对任意的$ t\in[0,1] $, 都有$ \lim\limits_{M\to\infty}\exp(-M(t-t^b))=0 $, 从而由Lebesgue控制收敛定理, 有
	\[
	\lim_{M\to\infty}\int_0^1\exp(-M(t-t^b))\diff t=0,
	\]
	于是可以取到充分大的$ M $使得
	\[
	\lambda=a\cdot\int_0^1\exp(-M(t-t^b))\diff t<1,
	\]
	此时
	\[
	\norm{Tf-Tg}\leqslant\lambda\cdot\norm{f-g},
	\]
	即$ T $是一个压缩映射.
	
	(3) 方程\,\eqref{eq:3.6题公式}\,成立当且仅当$ Tf=f $, 由压缩映照原理可知$ T $有唯一不动点, 即方程\,\eqref{eq:3.6题公式}\,有唯一解.\qed
	\end{Proof}
	
	\textbf{习题3.7}\ [习题课]\ \ 设$ E $是域$ \K $上的无限维线性空间, 设$ (e_i)_{i\in\alpha} $是$ E $中的一组向量. 若$ E $中任意向量可以用$ (e_i)_{i\in\alpha} $中的有限个向量唯一线性表示, 即对任意$ x\in E $, 存在唯一一组$ (x_i)_{i\in\alpha}\subset\K $使得仅有有限多个$ x_i\ne 0 $且$ x=\sum\limits_{i\in\alpha}x_ie_i $成立, 则称$ (e_i)_{i\in\alpha} $是$ E $中的Hamel基.
	\begin{enumerate}[(1)]
	\item 由Zorn引理证明$ E $有一组Hamel基.
	\item 假设$ E $还是一个赋范空间, 证明$ E $上必存在不连续的线性泛函.
	\item 证明: 在任意无限维赋范空间上, 一定存在一个比原来的范数严格强的范数(即新范数诱导的拓扑一定比原来的范数诱导的拓扑强且不相同), 由此说明若线性空间$ E $上任意两个范数都诱导相同的拓扑, 则$ E $有限维.
	\end{enumerate}
	\begin{Proof}
	(1) 记$ \mathcal G $是$ E $中线性无关集全体(线性无关集即其中元素线性无关, 在有限情形下与高等代数的定义相同, 在无限情形下要求任意有限子集均线性无关), 则$ (\mathcal G,\subseteq) $是一个偏序集. 任取$ \mathcal G $中的一个全序子集$ \mathcal E $, 并取$ D=\bigcup\mathcal E $, 再取$ F\in\fin D $. 那么对任意的$ x\in F $, 存在$ A(x)\in\mathcal E $使得$ x\in A(x) $, 注意到$ \{ A(x) : x\in F \} $是全序集, 取其最大元$ A(z) $, 则有$ F\subset A(z) $, 从而$ F $是线性无关集, 这说明$ D $也是线性无关集.
	
	因$ D $是线性无关集可知$ D\in\mathcal G $, 且$ D $是$ \mathcal E $的一个上界, 从而有Zorn引理可知$ \mathcal G $存在极大元$ T $, 那么$ T $是$ E $的Hamel基.
	
	(2) 设$ (e_i)_{i\in\alpha} $是$ E $的一组Hamel基, 任取$ e_1\in(e_i)_{i\in\alpha} $, 由$ \dim E\geqslant\aleph_0 $可知存在$ e_2\in(e_i)_{i\in\alpha} $使得$ \{ e_1,e_2 \} $线性无关, 且依此进行下去可以获得一线性无关的点列$ (e_n)_{n\geqslant 1} $, 并且可以用$ \frac{1}{n}\cdot\frac{e_n}{\norm{e_n}}=\mathrm{sgn}\,e_n/n $代替$ e_n $, 再记除上述$ (e_n)_{n\geqslant 1} $外其他的$ e_j $为$ (e_j)_{j\in\beta} $, 也即
	\[
	(e_i)_{i\in\alpha}=(e_n)_{n\geqslant 1}\cup(e_j)_{j\in\beta}.
	\]
	那么任取$ x\in E $, 有$ x=\sum\limits_{i\in\alpha}x_ie_i $, 其中除有限项外$ x_i=0 $, 再取映射
	\[
	f : E\mapsto \K,\qquad x\to\sum_{i\in\alpha}x_i,
	\]
	则$ f $是$ E $上的线性泛函, 且$ f(e_n)=1 $, 但$ \lim\limits_{n\to\infty}e_n=0 $, 从而$ f $不连续.
	
	(3) 记$ E $上原范数$ \norm{\cdot} $, 将(2)中找到的线性泛函记作$ f $, 并且取
	\[
	\norm{x}_1=\norm{x}+\abs{f(x)},\qquad \forall x\in E,
	\]
	容易证明$ \norm{\cdot}_1 $是$ E $上的范数, 且$ \norm{x}\leqslant\norm{x}_1 $, 从而$ \norm{\cdot}_1 $诱导的拓扑更强.(这因$ \id_E : (E,\norm{\cdot}_1)\to(E,\norm{\cdot}) $连续.)\qed
	\end{Proof}

	\textbf{习题3.8}\ [作业]\ \ 设$ E $是数域$ \K $上的有限维线性空间, 其维数$ \dim E=n $, $ \{ e_1,e_2,\cdots,e_n \} $是$ E $的一组基. 任取$ u\in\CL(E) $, 令$ [u] $表示$ u $在这组基下对应的矩阵
	\begin{enumerate}[(1)]
	\item 证明映射$ u\mapsto[u] $建立了$ \CL(E) $到$ \mathbb{M}_n(\K) $之间的同构映射.
	\item 假设$ E=\K^n $且$ \{ e_1,e_2,\cdots,e_n \} $是经典基(即$ e_i=[\delta_{i,1},\delta_{i,2},\cdots,\delta{i,n}] $, 这里$ \delta $是Kronecker符号), 并约定$ E=\K^n $上赋予Euclid范数. 证明: 若$ [u] $可被正交相似对角化, 则$ \norm{u}=\max\{ \abs{\lambda_1},\abs{\lambda_2},\cdots,\abs{\lambda_n} \} $, 这里$ \lambda_1,\lambda_2,\cdots,\lambda_n $是$ u $的特征值.
	\item 取$ \{ e_1,e_2,\cdots,e_n \} $如上, 试由$ [u] $中的元素分别确定$ p=1 $与$ p=\infty $时$ u : (\K^n,\norm{\cdot}_p)\to(\K^n,\norm{\cdot}_p) $的范数.
	\end{enumerate}
	\begin{Proof}
	(1) 由高等代数知识可知$ u\mapsto[u] $是一个线性的双射, 而任取$ \mathbb{M}_n(\K) $上的一个范数$ \norm{\cdot} $, 取
	\[
	\norm{u}=\norm{[u]}
	\]
	即是$ \CL(E) $上的一个范数, 这说明这一映射连续, 于是$ \CL(E) $与$ \mathbb{M}_n(\K) $同构.
	
	(2) 由$ u $可被正交相似对角化可知存在酉矩阵$ p $与对角阵$ \lambda=\mathrm{diag}\,\{ \lambda_1,\lambda_2,\cdots,\lambda_n \} $使得
	\[
	u=p^{-1}\lambda p.
	\]
	那么对任意单位向量$ x $, 记$ y=p^\dagger x $, 那么$ \norm{y}=\sqrt{x^\dagger pp^\dagger x}=\sqrt{x^\dagger x}=1 $. 从而
	\begin{align*}
	\norm{ux}=\sqrt{x^\dagger uu^\dagger x}=\sqrt{x^\dagger p\lambda^\dagger\lambda p^\dagger x}&=\sqrt{\sum_{k=1}^n\abs{\lambda_k}^2\abs{y_k}^2}\\
	&\leqslant\max_{k}\abs{\lambda_k}\sqrt{\sum_{k=1}^{n}\abs{y_k}^2}=\max_{k}\abs{\lambda_k}.
	\end{align*}
	于是$ \norm{u}\leqslant\max\limits_{k}\abs{\lambda_k} $. 而取$ k_0 $是使得特征值模最大的下标, 并设$ \lambda_{k_0} $对应的特征向量是$ x_{k_0} $, 那么由
	\[
	\norm{ux_{k_0}}=\abs{\lambda_{k_0}}\norm{x_{k_0}}=\max_{k}\abs{\lambda_k}\norm{x_{k_0}},
	\]
	从而$ \norm{u}=\max\limits_{k}\abs{\lambda_k} $.
	
	(3) 当$ p=1 $时, 对$ \forall x\in\K^n $, 有$ x=\sum\limits_{k=1}^n x_ke_k $成立, 那么由定义有
	\begin{align*}
	\norm{ux}_1=\norm{\begin{bmatrix}
	\Sigma u_{1,k}x_k\\\vdots\\\Sigma u_{n,k}x_k
	\end{bmatrix}}=\sum_{i=1}^n\abs{\sum_{k=1}^nu_{i,k}x_k}&\leqslant\sum_{i=1}^n\sum_{k=1}^n\abs{u_{i,k}}\abs{x_k}\\
	&\leqslant\max_{k}\left(\sum_{i=1}^n\abs{u_{i,k}}\right)\sum_{k=1}^n\abs{x_k}=\max_{k}\left(\sum_{i=1}^n\abs{u_{i,k}}\right)\norm{x}_1,
	\end{align*}
	从而$ \norm{u}_1\leqslant\max\limits_{k}\left(\sum\limits_{i=1}^n\abs{u_{i,k}}\right) $. 再取$ k_0 $是使得右侧取最大的列指标, 那么
	\[
	\norm{ue_{k_0}}_1=\max_{k}\left(\sum_{i=1}^n\abs{u_{i,k}}\right)\norm{e_{k_0}}_1,
	\]
	从而有$ \norm{u}_1=\max\limits_{k}\left(\sum\limits_{i=1}^n\abs{u_{i,k}}\right) $.
	
	而当$ p=\infty $时, 有
	\begin{align*}
	\norm{ux}_\infty=\max_i\abs{\sum_{k=1}^nu_{i,k}x_k}&\leqslant\max_i\sum_{k=1}^n\abs{u_{i,k}}\abs{x_k}\\
	&\leqslant\max_i\max_k\abs{x_k}\sum_{k=1}^n\abs{u_{i,k}}=\max_i\sum_{k=1}^n\abs{u_{i,k}}\norm{x}_\infty.
	\end{align*}
	从而$ \norm{u}_\infty\leqslant\max\limits_i\sum\limits_{k=1}^n\abs{u_{i,k}} $. 再取$ i_0 $是使得右侧最大的行指标, 那么
	\[
	\norm{u(\mathrm{sgn}\,u_{i_0,k}e_{i_0})}_\infty=\max_i\sum_{k=1}^n\abs{u_{i,k}}\norm{\mathrm{sgn}\,u_{i_0,k}e_{i_0}}_\infty,
	\]
	故$ \norm{u}_\infty=\max\limits_i\sum\limits_{k=1}^n\abs{u_{i,k}} $.\qed
	\end{Proof}
	
	\textbf{习题3.9}\ [作业]\ \ 设$ E $是Banach空间.
	\begin{enumerate}[(1)]
	\item 设$ u\in\CB(E) $且$ \norm{u}<1 $, 证明$ \id_E-u $在$ \CB(E) $中可逆;
	\item 设$ GL(E) $是$ \CB(E) $中可逆元构成的集合, 证明: $ GL(E) $关于复合运算构成一个群, 且它是$ \CB(E) $中的开集;
	\item 证明$ u\mapsto u^{-1} $是$ GL(E) $上的同胚.
	\end{enumerate}
	
	\begin{Proof}
	(1) 由$ \norm{u}<1 $和
	\[
	\sum_{n\geqslant 0}\norm{u^n}\leqslant\sum_{n\geqslant 0}\norm{u}^n<\infty
	\]
	可知级数$ \sum\limits_{n\geqslant 0}u^n $绝对收敛, 而$ \CB(E) $完备, 从而$ \sum\limits_{n\geqslant 0}u^n $收敛, 那么由
	\[
	(\id_E-u)\sum_{n\geqslant 0}u^n=\sum_{n\geqslant 0}u^n-\sum_{n\geqslant 1}u^n=\id_E
	\]
	可知$ \id_E-u $是右可逆的, 其右逆为$ \sum\limits_{n\geqslant 0}u^n $. 同理可证它是左可逆的, 且左逆也是$ \sum\limits_{n\geqslant 0}u^n $, 故$ \id_E-u $可逆, 其逆就是$ \sum\limits_{n\geqslant 0}u^n $.
	
	(2) 首先$ GL(E) $关于复合运算封闭, 且满足结合律, $ \id_E $是其中的单位元且由定义可知其中每个元素都有逆元. 于是$ (GL(E),\circ) $是一个群. 下面说明$ GL(E) $是$ \CB(E) $中的开集, 只需要说明对$ u\in GL(E) $, 对任意$ v $使得$ \norm{v-u}<\delta $, 都有$ v\in GL(E) $即可. 注意到
	\[
	v=u-(u-v)=u(\id_E-u^{-1}(u-v)),
	\]
	且$ \norm{v-u}<\delta<1/\norm{u^{-1}} $时, 有
	\[
	\norm{u^{-1}(u-v)}\leqslant \norm{u^{-1}}\norm{u-v}<1,
	\]
	由(1)的结论可知$ \id_E-u^{-1}(u-v) $可逆, 从而$ v\in GL(E) $, 也即$ GL(E) $是开集.
	
	(3) 容易证明$ u\mapsto u^{-1} $是到自身的双射, 只需要证明它连续. 注意到当$ \norm{u}<1 $时, 有
	\[
	\norm{(\id_E-u)^{-1}}=\norm{\sum_{n\geqslant 0}u^n}\leqslant\sum_{n\geqslant 0}\norm{u}^n\leqslant(1-\norm{u})^{-1}.
	\]
	于是对使得$ \norm{u-v} $足够小的$ v\in GL(E) $, 由
	\begin{align*}
	\norm{v^{-1}-u^{-1}}&=\norm{v^{-1}(u-v)u^{-1}}\\
	&\leqslant\norm{v^{-1}}\norm{u-v}\norm{u^{-1}}\\
	&=\norm{(\id_E-u^{-1}(u-v))^{-1}u^{-1}}\norm{u-v}\norm{u^{-1}}\\
	&\leqslant(1-\norm{u^{-1}(u-v)})^{-1}\norm{u-v}\norm{u^{-1}}^2\\
	&\leqslant(1-\norm{u^{-1}}\norm{u-v})^{-1}\norm{u-v}\norm{u^{-1}}^2\\
	&\leqslant 2\norm{u-v}\norm{u^{-1}}^2
	\end{align*}
	可知映射$ u\mapsto u^{-1} $是连续的. 同理可证其逆连续, 从而它是同胚.\qed
	\end{Proof}

	\textbf{习题3.10}\ [习题课]\ \ 设$ f\in L_2(\R) $, $ g(x)=\frac{1}{x}1_{[1,\infty)}(x) $, 证明: $ fg\in L_1(\R) $. 并举出反例说明$ f_1,f_2\in L_1(\R) $时$ f_1f_2\notin L_1(\R) $.
	\begin{Proof}
	由H\"older不等式可知
	\[
	\int_\R\abs{f(x)g(x)}\diff x\leqslant\left(\int_\R\abs{f(x)}^2\diff x\right)^{1/2}\left(\int_R\abs{g(x)}^2\diff x\right)^{1/2}=\norm{f}_2<\infty,
	\]
	从而$ fg\in L_1(\R) $.
	
	反例可取$ f_1=f_2=\frac{1}{\sqrt{x}}1_{(0,1]}(x) $, 则$ f_1,f_2\in L_1(\R) $, 但$ f_1f_2=\frac{1}{x}1_{(0,1]}(x)\notin L_1(R) $.\qed
	\end{Proof}
	
	\textbf{习题3.11}\ [习题课]\ \ 设$ (X,\CA, \mu) $是有限测度空间.
	\begin{enumerate}[(1)]
	\item 证明: 若$ 0<p<q\leqslant\infty $, 那么$ L_q(X)\subset L_p(\varOmega) $, 并举反例说明结论在$ \mu(X)=\infty $的测度空间上不成立.
	\item 证明: 若$ f\in L_\infty(X) $, 则$ f\in\bigcap_{p<\infty}L_p(X) $且$ \lim\limits_{p\to\infty}\norm{f}_p=\norm{f}_\infty $.
	\item 设$ f\in\bigcap_{p<\infty}L_p(X) $且满足$ \limsup\limits_{p\to\infty}\norm{f}_p<\infty $, 证明$ f\in L_\infty(X) $.
	\end{enumerate}
	\begin{Proof}
	(1) 若$ q=\infty $, 结论显然成立, 下设$ p<\infty $. 取$ p'=q/p $而$ q'=q/(q-p) $, 那么$ 1/p'+1/q'=1 $. 由H\"older不等式可知
	\[
	\begin{aligned}
	\norm{f}_p^p&=\int_X\abs{f}^p\diff\mu=\int_X\abs{f}^p\cdot 1\diff\mu=\norm{\abs{f}^p\cdot 1}_1\\
	&\leqslant\norm{\abs{f}^p}_{p'}\cdot\norm{1}_{q'}=\left(\int_X(\abs{f}^p)^{p'}\right)^{1/p'}\mu(X)^{1/q'}=\norm{f}_q^p\cdot\mu(X)^{1/q'}
	\end{aligned}
	\]
	即$ \norm{f}_p\leqslant\norm{f}_q\cdot\mu(X)^{1/p-1/q} $, 从而$ \norm{f}_q<\infty\Longrightarrow\norm{f}_p<\infty $, 即$ L_q(X)\subset L_p(X) $.
	
	若允许$ \mu(X)=\infty $, 取$ X=\R $, $ \mu $是Lebesgue测度, $ f=\frac{1}{x}1_{[1,\infty)} $即可. 此时$ f\in L_2(\R) $但$ f\notin L_1(R) $.
	
	(2) 任取$ f\in L_\infty(X) $, $ \forall p<\infty $都有
	\[
	\norm{f}_p=\left(\int_X\abs{f}^p\diff\mu\right)^{1/p}\leqslant\left(\int_X\norm{f}_\infty^p\diff\mu\right)^{1/p}=\norm{f}_\infty\cdot\mu(X)^{1/p}<\infty.
	\]
	即$ f\in L_p(X) $, 从而$ f\in\bigcap_{p<\infty}L_p(X) $. 再令$ p\to\infty $, 此时由$ \mu(X)^{1/p}\to 1 $可知
	\[
	\limsup_{p\to\infty}\norm{f}_p\leqslant\norm{f}_\infty.
	\]
	而$ \forall\varepsilon>0 $, 有$ \mu\{ \abs{f}>\norm{f}_\infty-\varepsilon \}>0 $, 记该集合为$ A $, 那么
	\[
	\norm{f}_p\geqslant\left(\int_A(\norm{f}_\infty-\varepsilon)^p\diff\mu\right)^{1/p}=(\norm{f}_\infty-\varepsilon)\mu(A)^{1/p}.
	\]
	令$ p\to\infty $有
	\[
	\liminf_{p\to\infty}\norm{f}_p\geqslant\norm{f}_\infty-\varepsilon,
	\]
	再令$ \varepsilon\to 0^+ $, 则$ \liminf\limits_{p\to\infty}\norm{f}_p\geqslant\norm{f}_\infty $, 则
	\[
	\norm{f}_\infty\leqslant\liminf_{p\to\infty}\norm{f}_p\leqslant\limsup_{p\to\infty}\norm{f}_p\leqslant\norm{f}_\infty.
	\]
	也即$ \lim\limits_{p\to\infty}\norm{f}_p=\norm{f}_\infty $.
	
	(3) 用反证法, 若$ f\notin L_\infty(X) $, 则$ \forall n\in\N $, 都有$ \mu\{ \abs{f}\geqslant n \}>0 $, 设$ B=\{ \abs{f}\geqslant 0 \} $, 则有
	\[
	\norm{f}_p\geqslant\left(\int_B\abs{f}^p\diff\mu\right)^{1/p}=n\cdot\mu(B)^{1/p}.
	\]
	取上极限
	\[
	\limsup_{p\to\infty}\norm{f}_p\geqslant\limsup_{p\to\infty}n\cdot\mu(B)^{1/p}\geqslant n,
	\]
	即$ \limsup\limits_{p\to\infty}\norm{f}_p=\infty $, 矛盾. 从而$ f\in L_\infty(X) $.\qed
	\end{Proof}

	\textbf{习题3.12}\ [作业]\ \ 设$ 0<p<q\leqslant\infty $, $ 0\leqslant\theta\leqslant 1 $, 并令
	\[
	\frac{1}{s}=\frac{\theta}{p}+\frac{1-\theta}{q},
	\]
	证明$ f\in L_p(X)\cap L_q(X) $, 则$ f\in L_s(X) $且
	\[
	\norm{f}_s\leqslant\norm{f}_p^{\theta}\norm{f}_q^{1-\theta}.
	\]
	\begin{Proof}
	取$ p'=p/\theta, q'=q/(1-\theta) $, 从而由$ 1/s=1/p'+1/q' $可知$ (p',q') $是一对共轭指数, 由H\"older不等式可知
	\[
	\norm{f}_s=\norm{f^\theta f^{1-\theta}}_s\leqslant\norm{f^\theta}_{p'}\norm{f^{1-\theta}}_{q'}=\norm{f}_p^\theta\norm{f}_q^{1-\theta}.
	\]
	且由$ f\in L_p(X)\cap L_q(X) $知$ \norm{f}_s<\infty $, 也即$ f\in L_s(X) $.\qed
	\end{Proof}

	\textbf{习题3.16}\ [作业]\ \ 在实数集$ \R $上取Lebesgue $ \sigma $-代数与Lebesgue测度, 并设$ f, g\in L_1(\R) $.
	\begin{enumerate}[(1)]
	\item 证明:
	\[
	\int_{\R\times\R}f(u)g(v)\diff u\diff v=\left(\int_\R f(u)\diff u\right)\left(\int_\R g(v)\diff v\right)=\int_\R\left(\int_\R f(x-y)g(y)\diff y \right)\diff x,
	\]
	并由此导出函数$ x\mapsto\int_\R f(x-y)g(y)\diff y $在$ \R $上几乎处处有定义.
	\item 定义$ f $与$ g $的\textbf{卷积}$ f\ast g $为
	\[
	f\ast g(x)=\begin{cases}
	\int_\R f(x-y)g(y)\diff y & ,\text{当积分存在时}\\
	0 & ,\text{其他情形}
	\end{cases}
	\]
	证明$ f\ast g\in L_1(\R) $且$ \norm{f\ast g}_1\leqslant\norm{f}_1\norm{g}_1 $.
	\item 取$ f=1_{[0,1]} $, 求$ f\ast f $.
	\end{enumerate}
	\begin{Proof}
	(1) 由Fubini定理可知
	\[
	\begin{aligned}
	\int_{\R\times\R} f(u)g(v)\diff u\diff v&=\int_\R\left(\int_\R f(u)g(v)\diff v\right)\diff u\\&=\int_\R g(v)\left(\int_\R f(u)\diff u\right)\diff v=\left(\int_\R f(u)\diff u\right)\left(\int_\R g(v)\diff v\right),
	\end{aligned}
	\]
	且
	\[
	\begin{aligned}
	\int_\R\left(\int_\R f(x-y)g(y)\diff y \right)\diff x&=\int_\R g(y)\left(\int_\R f(x-y)\diff x\right)\diff y\\
	&=\int_\R g(y)\left(\int_\R f(x-y)\diff (x-y)\right)\diff y\\
	&=\left(\int_\R f(u)\diff u\right)\left(\int_\R g(v)\diff v\right).
	\end{aligned}
	\]
	于是所求证等式成立. 其中右半侧使用了Lebesgue测度的平移不变性. 另外由于
	\[
	\begin{aligned}
	\abs{\int_\R\left(\int_\R f(x-y)g(y)\diff y \right)\diff x}&\leqslant\abs{\left(\int_\R f(u)\diff u\right)\left(\int_\R g(v)\diff v\right)}\\&=\abs{\int_\R f(u)\diff u}\abs{\int_\R g(v)\diff v}<\infty,
	\end{aligned}
	\]
	于是函数$ x\mapsto\int_\R f(x-y)g(y)\diff y $在$ \R $上几乎处处有定义.
	
	(2) 由于
	\[
	\abs{f\ast g(x)}=\abs{\int_\R f(x-y)g(y)\diff y}\leqslant\int_\R\abs{f(x-y)}\abs{g(y)}\diff y,
	\]
	两侧同时在$ \R $上对$ x $积分可知
	\[
	\int_\R\abs{f\ast g(x)}\diff x\leqslant\int_\R\int_\R\abs{f(x-y)}\abs{g(y)}\diff y\diff x=\int_\R\abs{f(u)}\diff u\int_\R\abs{g(v)}\diff v,
	\]
	此即$ \norm{f\ast g}_1\leqslant\norm{f}_1\norm{g}_1 $, 且由$ \norm{f\ast g}<\infty $可知$ f\ast g\in L_1(\R) $.
	
	(3) 注意到$ 1_{[0,1]}(x-t)=1_{[x-1,x]}(t) $, 从而由卷积的定义, 有
	\[
	\begin{aligned}
	f\ast f(x)=\int_\R f(x-y)f(y)\diff y&=\int_\R 1_{[0,1]}(x-y)1_{[0,1]}(y)\diff y\\&=\int_\R 1_{[x-1,x]}(y)1_{[0,1]}(y)\diff y=m([x-1,x]\cap[0,1]),
	\end{aligned}
	\]
	于是
	\[
	f\ast f(x)=\begin{cases}
	0 &,x<0\\x &,0\leqslant x\leqslant 1\\2-x&,1\leqslant x\leqslant 2\\0 &,x>2.
	\end{cases}
	\]
	\qed
	\end{Proof}

\section{第4章习题}
	
\textbf{习题4.2}\ [作业]\ \ 设 $ A $ 是 $ \ell_{2} $ 的子集, 其元素 $ x = (x_{n})_{n\geqslant1} $ 满足 $ \abs{x_{n}}\leqslant1/n, n\geqslant1 $. 证明 $ A $ 是紧集. 
	\begin{Proof}
		$ \ell_{2} $ 是 Hilbert空间且易证 $ A $ 是闭集, 故 $ A $ 完备. 为证 $ A $ 是紧集, 只需证 $ A $ 是预紧的. 

		因为对任意的 $ \varepsilon>0 $, 存在 $ n_{0}\in\N $ 使得
		\[
			\begin{aligned}
				\sum_{k\geqslant n_{0}+1}\abs{x_{k}}^{2} & \leqslant \frac{1}{(n_{0}+1)^{2}}+\frac{1}{(n_{0}+2)^{2}}+\cdots\\
				& \leqslant \frac{1}{n_{0}(n_{0}+1)}+\frac{1}{(n_{0}+1)(n_{0}+2)}+\cdots\\
				& = \frac{1}{n_{0}}<\frac{\varepsilon^{2}}{2}. 
			\end{aligned}
		\]
		对上述的 $ n_{0} $, 令 $ F=\{ (\seq[n_{0}]{x}, 0, \dots, 0, \dots) : x_{i}\in\C, 1\leqslant i\leqslant n_{0} \}\subset\ell_{2} $. 则 $ F $ 有限维, 再令
		\[
			B = \left\{ x= (\seq[n_{0}]{x}, 0,\dots, 0\dots) : \abs{x_{i}}\leqslant\frac{1}{i} \right\}\subset A, 
		\]
		取 $ B $ 中元素的前 $ n_{0} $ 项构成 $ \tilde{B} $, 即
		\[
			\tilde{B}=\left\{ x= (\seq[n_{0}]{x}) : \abs{x_{i}}\leqslant\frac{1}{i} \right\}
		\]
		显然 $ B $ 与 $ \tilde{B} $ 同构. 且 $ B $ 在 $ F $ 中闭, 又 $ \abs{x_{1}}^{2}+\cdots+\abs{x_{n_{0}}}^{2}\leqslant1+1=2 $, 故 $ B $ 在 $ F $ 中有界闭, 从而紧, 所以有 $ \tilde{B} $ 在 $ \C^{n_{0}} $ 中紧, 于是 $ \tilde{B} $ 预紧, 即存在 $ \tilde{B} $ 中有限个点 $ y^{(1)}, y^{(2)}, \dots, y^{(k)} $, 其中
		\[
			y^{(i)}=(y^{(i)}_{1}, \dots, y^{(i)}_{n_{0}} ), \quad i = 1, 2, \dots, k
		\]
		使得 $ \tilde{B}\subset\bigcup_{j = 1}^{k}B(y^{(j)}, \varepsilon/\sqrt{2}) $, 则 $ A\subset \bigcup_{j = 1}^{k}B(\tilde{y}^{(j)}, \varepsilon) $, 其中 $ \tilde{y}^{(i)}=(\seq[n_{0}]{y^{(i)}}, 0, \dots) $, 故 $ A $ 预紧, 则 $ A $ 紧. \qed
	\end{Proof}

	\textbf{习题4.6}\ [作业]\ \ 设 $ H $ 是内积空间,  $ x_{n}, x\in H $, 并假设
	\[
		\lim_{n\to\infty}\norm{x_{n}}=\norm{x}\ \ \text{且}\ \  \lim_{n\to\infty}\lrangle{y, x_{n}}=\lrangle{y, x},\quad y\in H
	\]
	证明 $ \lim\limits_{n\to\infty}\norm{x_{n}-x}=0 $.
	\begin{Proof}
		因为
		\[
			\norm{x_{n}-x}^{2}=\norm{x_{n}}^{2}+\norm{x}^{2}-2\Re\lrangle{x_{n}, x}
		\]
		令 $ n\to\infty $, 有
		\[
			\lim_{n\to\infty}\norm{x_{n}-x}^{2}=\norm{x}^{2}+\norm{x}^{2}-2\Re\lrangle{x, x} = 0, 
		\]
		命题得证. \qed
	\end{Proof}

	\textbf{习题4.10}\ [作业]\ \ 证明以下命题:
	\begin{enumerate}[(1)]
		\item 设 $ H $ 是 Hilbert空间, $ D_{n}=\{ -1, 1 \}^{n} $. 证明
		\[
			\frac{1}{2^{n}}\sum_{(\varepsilon_{k})\in D_{n}}\norm{\varepsilon_{1}x_{1}+\cdots+\varepsilon_{n}x_{n}}^{2}=\norm{x_{1}}^{2}+\cdots+\norm{x_{n}}^{2}, \quad \forall \seq{x}\in H
		\]
  		\item 设 $ (X, \norm{\cdot}) $ 是Banach空间, 并假设有一个 $ X $ 上的内积范数 $ \abs{\cdot} $ 等价于 $ \norm{\cdot} $. 证明存在正常数 $ a $ 和 $ b $, 使得
  		\[
			a\sum_{k=1}^{n}\norm{x_{k}}^{2}\leqslant\frac{1}{2^{n}}\sum_{(\varepsilon_{k})\in D_{n}}\norm{\sum_{k=1}^{n}\varepsilon_{k}x_{k}}^{2}\leqslant b\sum_{k=1}^{n}\norm{x_{k}}^{2}, \quad \forall x_{1}, x_{2}, \dots, x_{n} \in X.
		\]
		\item 设 $ 1\leqslant p\ne2\leqslant\infty $ 证明空间 $ c_{0}, \ell_{p}, L_{p} $ 中没有等价的内积范数.
	\end{enumerate}
	\begin{Proof}
		(1) 因为
		\[
			\frac{1}{2^{n}}\sum_{(\varepsilon_{k})\in D_{n}}\norm{\varepsilon_{1}x_{1}+\cdots+\varepsilon_{n}x_{n}}^{2}=\frac{2^{n}}{2^{n}}(\norm{x_{1}}^{2}+\cdots+\norm{x_{n}}^{2})+\frac{1}{2^{n}}\sum_{(\varepsilon_{k})\in D_{k}}\sum_{i\ne j}\varepsilon_{i}\varepsilon_{j}\lrangle{x_{i}, x_{j}},
		\]
		由对称性可知 
		\[
			\frac{1}{2^{n}}\sum_{(\varepsilon_{k})\in D_{k}}\sum_{i\ne j}\varepsilon_{i}\varepsilon_{j}\lrangle{x_{i}, x_{j}}=0
		\]
		命题得证. 

		(2) 设内积范数 $ \abs{\cdot} $ 等价于 $ \norm{\cdot} $, 则存在常数 $ m, M>0 $ 使得
		\[
			m\abs{x}\leqslant\norm{x}\leqslant M\abs{x}, \quad x\in H,
		\]
		故 
		\[
			\begin{aligned}
				\frac{1}{2^{n}}\sum_{(\varepsilon_{k})\in D_{n}}\norm{\sum_{k=1}^{n}\varepsilon_{k}x_{k}}^{2} & \leqslant \frac{M^{2}}{2^{n}}\sum_{(\varepsilon_{k})\in D_{k}}\abs{\sum_{k=1}^{n}\varepsilon_{k}x_{k}}^{2}  \\
				& = M^{2}\left(\sum_{k=1}^{n}\abs{x_{k}}^{2}\right)\leqslant \frac{M^{2}}{m^{2}}\left(\sum_{k=1}^{n}\norm{x_{k}}^{2}\right)
			\end{aligned}
		\]
		同理有
		\[
			\begin{aligned}
				\frac{1}{2^{n}}\sum_{\left(\varepsilon_{k}\right)\in D_{n}}\norm{\sum_{k=1}^{n}\varepsilon_{k}x_{k}}^{2} & \geqslant \frac{m^{2}}{2^{n}}\sum_{\left(\varepsilon_{k}\right)\in D_{k}}\abs{\sum_{k=1}^{n}\varepsilon_{k}x_{k}}^{2}  \\
				 & = m^{2}\left(\sum_{k=1}^{n}\abs{x_{k}}^{2}\right)\geqslant \frac{m^{2}}{M^{2}}\left(\sum_{k=1}^{n}\norm{x_{k}}^{2}\right)
			\end{aligned}
		\]
		取 $ a=m^{2}/M^{2}, b=M^{2}/m^{2} $ 即可. 

		(3) 对 $ c_{0} $. $ \forall n\geqslant1 $, 考虑 $ \seq{e} $, 则 $ \norm{e_{k}}_{\infty}=1 $, 且 $ \norm{\sum\limits_{k=1}^{n}\varepsilon_{k}e_{k}}_{\infty}^{2}=1 $, 故若 $ \norm{\cdot}_{\infty} $ 有等价的内积范数 $ \abs{\cdot} $, 则有
		\[
			an\leqslant1\leqslant bn, \quad n\in\N
		\]
		矛盾. $ \ell_{\infty} $ 与 $ c_{0} $ 同理. 

		对 $ \ell_{p}\,(1\leqslant p<\infty) $. 考虑 $ \seq{e} $, 则 $ \norm{e_{k}}_{p}=1 $, 且 $ \norm{\sum\limits_{k=1}^{n}\varepsilon_{k}e_{k}}^{2}_{p}=n^{2/p} $, 若 $ \norm{\cdot}_{p} $ 有等价的内积范数 $ \abs{\cdot} $, 有
		\[
			an\leqslant n^{2/p} \leqslant bn\Longrightarrow a\leqslant n^{2/p-1}\leqslant b, \quad n\in\N,
		\]
		因为 $ p\ne 2 $, 所以矛盾. 

		而对 $ L_{p}[0, 1], (1\leqslant p\leqslant\infty) $, 取
		\[
			x_{k}=n\cdot1_{[{k}/{n^p}, ({k+1})/{n^p})}, \quad k = 0, 1, \dots, n-1,
		\]
		则 $ \norm{x_{k}}_{p}=1 $ 且 $ \norm{\sum\limits_{k=1}^{n}\varepsilon_{k}x_{k}}_{p}^{2}=n^{2/p} $, 与 $ \ell_{p} $ 同理可证. \qed
	\end{Proof}

	\textbf{习题4.11}\ [作业]\ \ 设 $ (C_{n})_{n\geqslant1} $ 是 Hilbert空间 $ H $ 中的一个递增非空闭凸子集列,  $ C $ 是所有 $ C_{n} $ 并集的闭包, 证明
	\[
		P_{C}(x) = \lim_{n\to \infty}P_{C_{n}}(x), \quad \forall x\in H.
	\]
	\begin{Proof}
		先证 $ P_{C_{n}}(x) $ 收敛. 因为 $ C_{n} $ 递增, 故 $ (\norm{x-P_{C_{n}}(x)})_{n\geqslant1} $ 单调递减有下界, 从而它有极限, 对 $ n<m $, 有
		\[
			\begin{aligned}
				\norm{P_{C_{n}}(x)-P_{C_{m}}(x)}^{2} & =\norm{P_{C_{n}}(x)-x+x-P_{C_{m}}(x)}^{2} \\
				& = 2(\norm{x-P_{C_{n}}(x)}^{2}+\norm{x-P_{C_{m}}(x)}^{2})-\norm{P_{C_{m}}(x)+P_{C_{n}}(x)-2x}^{2} \\
				& = 2(\norm{x-P_{C_{n}}(x)}^{2}+\norm{x-P_{C_{m}}(x)}^{2})-4\norm{x-\frac{P_{C_{n}}(x)+P_{C_{m}}(x)}{2}}^{2} \\
				& \leqslant 2(\norm{x-P_{C_{n}}(x)}^{2}+\norm{x-P_{C_{m}}(x)}^{2}) - 4\norm{x-P_{C_{m}}(x)}^{2}\\
				& = 2(\norm{P_{C_{n}}(x)-x}^{2}-\norm{P_{C_{m}}(x)-x}^{2})\to 0,\quad (n,m\to \infty)
			\end{aligned}	
		\]
		其中的不等号可以由闭凸集投影定理~\ref{thm:闭凸集投影定理}~得出. 所以有 $ (P_{C_{n}}(x))_{n\geqslant1} $ 是 $ H $ 中的 Cauchy列, 故它收敛, 记其极限为 $ y $, 则由
		\[
			\forall n\geqslant1\,\forall z\in C_{n}\subset C\,(\Re\lrangle{x-P_{C_{n}}(x), z-P_{C_{n}}(x)}\leqslant0)
		\]
		与内积的连续性知
		\[
			\forall z\in C\,(\Re\lrangle{x-y, z-y}\leqslant0),\quad n\to\infty
		\]
		即 $ y = P_{C}(x) $, 从而  $ P_{C}(x)=\lim\limits_{n\to\infty}P_{C_{n}}(x) $.\qed
	\end{Proof}

	\textbf{习题4.13}\ [作业]\ \ 设$ E=C[0,1] $上赋有以下内积
	\[
	\lrangle{f,g}=\int_0^1f(t)\baro{g(t)}\diff t,
	\]
	并设$ E_0 $表示在$ [0,1] $上积分为0的函数组成的$ E $的线性子空间. 考虑$ E $的线性子空间$ H=\{ f\in E : f(1)=0 \} $且$ H_0=E_0\cap H $.
	\begin{enumerate}[(1)]
	\item 验证$ H_0 $是$ H $的闭真线性子空间.
	\item 设$ h(t)=t-1/2 $, $ t\in[0,1] $, 证明:
	\begin{enumerate}[(i)]
	\item $ E=\Span\{H,h\} $且$ E_0=\Span\{H_0,h\} $;
	\item $ h $属于$ H_0 $在$ E $中的闭包.
	\end{enumerate}
	\item 以$ H $为全空间, 证明$ H_0^\bot=\{0\} $, 并解释所得结果蕴涵的意义.
	\end{enumerate}
	\begin{Proof}
	(1) 由于$ E_0 $与$ H $都是线性空间, 故$ E_0\cap H $也是线性空间. 不妨设$ (f_n)_{n\geqslant 1}\subset E_0 $, 且存在$ f\in E $使得$ \norm{f_n-f}\to 0 $, 则由Cauchy-Schwarz不等式, 有
	\[
	\abs{\int_0^1f(t)\diff t}\leqslant\abs{\int_0^1(f-f_n)\diff t}+\abs{\int_0^1f_n\diff t}\leqslant\norm{f-f_n}\to 0.
	\]
	即$ f\in E_0 $. 从而$ E_0 $是闭的. 而取$ h(t)=1-t $即得$ H_0\ne H $.
	
	(2) (i) 对任意$ f\in E $, 注意到$ f-2f(1)h\in H $, 从而$ E=\Span\{ H,h \} $. 而对任意$ f\in E_0 $, 注意到$ f-2f(1)h\in H_0 $可知$ E_0=\Span\{ H_0,h \} $.
	
	(ii) 设$ f_n $是连接$ (0,0),\ (1/n,h(1/n)), (1-1/n,h(1-1/n)), (1,0) $的分段线性函数, 那么由于$ \abs{h-f_n}\leqslant 1 $可知
	\[
	\abs{h(t)-f_n(t)}^2\leqslant\abs{h(t)-f_n(t)},\quad\forall t\in [0,1]
	\]
	于是
	\[
	\int_0^1\abs{h-f_n}^2\diff t\leqslant\int_0^1\abs{h-f_n}\diff t=2\cdot\frac{1}{2}\cdot\frac{1}{n}\cdot\frac{1}{2}=\frac{1}{2n}\to 0\qquad (n\to\infty)
	\]
	于是$ \norm{f_n-h}\to 0 $, 且$ f_n\in H_0 $, 这说明$ h $属于$ H_0 $在$ E $中的闭包.
	
	(3) 当$ H $是全空间时, 由(2)可知$ E_0\subset\bar{H}_0 $, 从而任取$ f\in H_0^\bot $, 取$ g=f-\int_0^1f(t)\diff t $, 那么有$ g\in E_0\subset \bar{H}_0 $, 由$ f\in H_0^\bot $可知
	\[
	\lrangle{f,g}=\lrangle{f,f}-\abs{\int_0^1f(t)\diff t}^2=\norm{f}^2-\abs{\lrangle{f,1}}^2=0,
	\]
	也即$ \norm{f}^2\cdot\norm{1}^2=\abs{\lrangle{f,1}}^2 $, 注意到这就是Cauchy-Schwarz不等式取等, 于是由取等条件可知$ \exists\lambda\in\K $使得$ f=\lambda\cdot 1 $. 又因为$ f(1)=0 $, 从而$ \lambda=0 $, 也即$ f=0 $.
	
	本题说明当内积空间$ H $不完备时, 正交分解定理未必成立. 事实上, 若此时正交分解定理成立, 那么
	\[
	H=\bar{H}_0\oplus H_0^\bot=\bar{H}_0=H_0,
	\]
	而这与(1)矛盾.\qed
	\end{Proof}

	\textbf{习题4.15}\ [作业]\ \ 证明下列命题:
	\begin{enumerate}[(1)]
	\item 设$ E $和$ F $是Hilbert空间$ H $的两个正交线性子空间, 证明$ E+F $是闭的当且仅当$ E, F $都是闭的.
	\item 以$ (e_n)_{n\geqslant 1} $表示$ \ell_2 $中的标准正交基, 并设$ E $是$ \{ e_{2n} : n\geqslant 1 \} $的线性扩张的闭包, 而$ F $是$ \{ e_{2n}+\frac{1}{n}e_{2n+1} : n\geqslant 1 \} $的线性扩张的闭包. 证明$ E\cap F=\{0\} $且$ E+F $在$ \ell_2 $中不是闭的.
	\end{enumerate}
	\begin{Proof}
	(1) \textsl{必要性.}\ \ 设$ E+F $是闭的, 若有序列$ (x_n)_{n\geqslant 1}\subset E $使得$ x_n\to z $, 注意到$ x_n=x_n+0\in E+F $, 那么由$ E+F $的闭性可知$ z\in E+F $. 再设$ z=x+y $, 其中$ x\in E,\ y\in F $, 那么
	\[
	\norm{x_{n}-(x+y)}^2=\norm{x_{n}-x}^2+\norm{y}^2\to 0,\qquad n\to\infty.
	\]
	这说明$ \norm{y}^2=0 $, 也即$ y=0 $. 于是$ z\in E $, 这说明$ E $是闭的. 同理可证$ F $也是闭的.
	
	(2) \textsl{充分性.}\ \ 设$ E $和$ F $都是闭集, 若有序列$ (z_n)_{n\geqslant 1}=(x_{n}+y_{n})_{n\geqslant1}\subset E+F $, 其中$ x_n\in E,\ y_n\in F $, 且$ z_n\to z $, 只需证明$ z\in E+F $即可. 由$ (z_n)_{n\geqslant 1} $收敛可知
	\[
	\begin{aligned}
	\norm{x_n-x_m}^2&=\norm{(x_n+y_n)-y_n-(x_m+y_m)+y_m}^2\\
	&=\norm{(x_n+y_n)-(x_m+y_m)-(y_n-y_m)}^2\\
	&=\norm{z_n-z_m}^2+\norm{y_n-y_m}^2-2\Re\lrangle{z_n-z_m,y_n-y_m}\\
	&=\norm{z_n-z_m}^2-\norm{y_n-y_m}^2
	\end{aligned}
	\]
	也即$ \norm{x_n-x_m}^2+\norm{y_n-y_m}^2=\norm{z_n-z_m}^2\to 0 $, 即$ (x_n)_{n\geqslant 1} $和$ (y_n)_{n\geqslant 1} $分别是$ E $和$ F $上的Cauchy列. 由$ E $和$ F $的闭性可知存在$ x\in E,\ y\in F $使得$ x_n\to x,\ y_n\to y $, 从而$ z=x+y\in E+F $, 即$ E+F $是闭的.
	
	(2) 任取$ E, F $中的序列, 它们必是这样的形式:
	\[
	\begin{aligned}
	x & = (0,x_2,0,x_4,\dots,0,x_{2n},\dots)\\
	y & = (0,y_2,y_2,y_4,y_4/2,\dots,y_{2n},y_{2n}/n,\dots)
	\end{aligned}
	\]
	故若$ x=y $, 必有各$ y_n=0 $, 即$ E\cap F=\{0\} $.
	
	再证$ E+F $在$ \ell_2 $中不是闭集. 取$ E $中的序列$ a_n=-\sum\limits_{k=1}^ne_{2k} $和$ F $中的序列$ b_n=\sum\limits_{k=1}^n(e_{2k}+e_{2k+1}/k) $, 那么
	\[
	a_n+b_n=\sum_{k=1}^n\frac{e_{2k+1}}{k}\in E+F,
	\]
	考虑序列$ (a_n+b_n)_{n\geqslant 1} $的极限$ c $, 若$ c=\sum\limits_{k\geqslant 1}e_{2k+1}/k\in E+F $, 那么存在$ x\in E $与$ y\in F $使得
	\[
	\begin{aligned}
	\sum_{k\geqslant 1}\frac{e_{2k+1}}{k}=x+y&=\sum_{k\geqslant 1}x_{2k}e_{2k}+\sum_{k\geqslant 1}y_{2k}e_{2k}+\frac{y_{2k}e_{2k+1}}{k}\\
	&=\sum_{k\geqslant 1}(x_{2k}+y_{2k})e_{2k}+\sum_{k\geqslant 1}\frac{y_{2k}e_{2k+1}}{k},
	\end{aligned}
	\]
	这说明$ x_{2k}=-1 $且$ y_{2k}=1 $, 于是$ \sum\limits_{k\geqslant 1}\abs{x_k}^2=\infty $, 即$ x\notin E $, 也即$ c\notin E+F $. 故$ E+F $不是$ \ell_2 $中的闭集.\qed
	\end{Proof}

	\textbf{习题4.19}\ [习题课]\ \ 令 $ H=L_{2}(0, 1) $; 并设 $ (e_{n})_{n\geqslant1} $ 是 $ H $ 中的规范正交集, 证明 $ (e_{n})_{n\geqslant1} $ 是 $ H $ 上的规范正交基的充分必要条件是
	\[
		\sum_{n\geqslant1}\abs{\int_{0}^{x}e_{n}(t)\diff t}^{2} = x, \quad \forall x\in[0, 1].
	\]
	\begin{Proof}
		\textsl{必要性}. 设 $ (e_{n})_{n\geqslant1} $ 是 $ H $ 是规范正交基, 则 $ \forall x\in[0, 1] $, 都有 $ 1_{[0, x]}\in L_{2}[0, 1] $, 由 Parseval 恒等式有
		\[
			\begin{aligned}
				x = \norm{1_{[0, x]}}^{2} & =\sum_{n\geqslant1}\abs{\lrangle{1_{[0, x]}, e_{n}}}^{2}\\
				& =\sum_{n\geqslant1}\abs{\int_{0}^{1}e_{n}(t)\baro{1_{[0, x]}(t)}\diff t}^{2}=\sum_{n\geqslant1}\abs{\int_{0}^{x}e_{n}(t)\diff t}^{2}.
			\end{aligned}
		\]

		\textsl{充分性}. $ \forall x\in[0, 1] $, 由
		\[
			\begin{aligned}
				\norm{1_{[0, x]}-\sum_{k=1}^{n}\lrangle{1_{[0, x]}, e_{k}}e_{k}}^{2} & = x+\sum_{k=1}^{n}\abs{\lrangle{1_{[0, x]}, e_{k}}}^{2}-2\sum_{k=1}^{n}\abs{\lrangle{1_{[0, x]}, e_{k}}}^{2} \\
			& = x-\sum_{k=1}^{n}\abs{\lrangle{1_{[0, x]}, e_{k}}}^{2}\to 0,\qquad n\to\infty,
			\end{aligned}
		\]
		知 $ 1_{[0, x]} $ 可由 $ \Span\{ (e_{n})_{n\geqslant1} \} $ 中元素逼近. 则 $ \forall a<b $, $ a, b\in[0, 1] $, 由
		\[
			1_{[a, b]} = 1_{[0, b]}-1_{[0, a]}
		\]
		可知阶梯函数可以由 $ \Span\{ (e_{n})_{n\geqslant1} \} $ 中元素逼近, 而 $ L_{2}[0, 1] $ 中元素可由阶梯函数逼近, 故
		\[
			\baro{\Span\{ (e_{n})_{n\geqslant1} \}}=L_{2}[0, 1]
		\]
		从而 $ (e_{n})_{n\geqslant1} $ 是 $ L_{2}[0, 1] $ 的规范正交基. \qed
	\end{Proof}	

\section{第5章习题}

	\textbf{习题5.1}\ [习题课]\ \ 对任意$ x\in[0,1] $, 设$ f_n(x)=x^n $. 在$ [0,1] $的哪些点处, $ (f_n)_{n\geqslant 1} $等度连续?
	\begin{Proof}
	对任意的$ a\in[0,1) $, 有
	\[
	\abs{x^n-a^n}=\abs{(x-a)(x^{n-1}+x^{n-2}a+\cdots+a^{n-1})}\leqslant(1+a+\cdots+a^{n-1})\abs{x-a}<\frac{\abs{x-a}}{1-a},
	\]
	故$ (f_n)_{n\geqslant 1} $在$ x\in[0,1) $上等度连续. 但$ a=1 $时, 因为对任意的$ x\in(1-\delta,1) $, 有
	\[
	\lim_{n\to\infty}\abs{f_n(x)-f_n(1)}=1,
	\]
	从而存在$ n_0 $使得$ n\geqslant n_0 $时有
	\[
	\abs{f_{n_0}(x)-f_{n_0}(1)}>\frac{1}{2},
	\]
	于是$ (f_n)_{n\geqslant 1} $在$ x=1 $处不等度连续.\qed
	\end{Proof}

	\textbf{习题5.2}\ [作业]\ \ 设$ K $是度量空间而$ E $是赋范空间, $ (f_n)_{n\geqslant 1}\subset C(K,E) $.
	\begin{enumerate}[(1)]
	\item 证明: 若$ (f_n)_{n\geqslant 1} $在某一点$ x $等度连续, 那么对任意收敛到$ x $的点列$ (x_n)_{n\geqslant 1} $, 都有$ (f_n(x)-f_n(x_n))_{n\geqslant 1} $收敛到0.
	\item 进而证明: 若$ (f_n(x))_{n\geqslant 1} $在$ E $中收敛到$ y $, 那么对任意收敛到$ x $的点列$ (x_n)_{n\geqslant 1} $, 都有$ (f_n(x_n))_{n\geqslant 1} $也收敛到$ y $.
	\item 取$ f_n=\sin nx $, 证明$ (f_n)_{n\geqslant 1} $在$ \R $上无处等度连续.
	\end{enumerate}
	\begin{Proof}
	(1) 由$ (f_n)_{n\geqslant 1} $在$ x $处等度连续, 有
	\[
	\forall\varepsilon>0\,\exists\delta>0\,(d_K(y,x)<\delta\Rightarrow \norm{f_n(y)-f_n(x)}<\varepsilon),
	\]
	设$ (x_n)_{n\geqslant 1} $收敛于$ x $, 那么对上述$ \delta>0 $,
	\[
	\exists n_0\in\N\,(n\geqslant n_0\Rightarrow d_K(x_n,x)<\delta)
	\]
	即$ \norm{f_n(x_n)-f_n(x)}\leqslant\varepsilon $.
	
	(2) 进一步, 设$ (f_n(x))_{n\geqslant 1}\to y $, 那么
	\[
	\norm{f_n(x)-y}\leqslant\norm{f_n(x_n)-f_n(x)}+\norm{f_n(x)-y},
	\]
	由(1)知右侧收敛到0, 从而结论成立.
	
	(3) 设$ f_n(x)=\sin nx $, 那么$ x\ne k\pi, k\in\Z $时取$ x_n=x+\pi/n $, 则
	\[
	\abs{f_n(x)-f_n(x_n)}=\abs{\sin nx-\sin(nx+\pi)}=2\abs{\sin nx}\not\to 0,
	\]
	用反证法, 若$ \sin nx\to 0 $, 那么
	\[
	\lim_{n\to\infty}2\sin x\cos nx=\lim_{n\to\infty}\sin(n+1)x-\sin(n-1)x=0,
	\]
	又因为$ \sin x\ne 0 $, 从而$ \cos nx\to 0 $. 再由$ \sin^2 nx+\cos^2 nx=1 $, 从而$ \sin nx\to 1 $, 矛盾. 从而$ (\sin nx)_{n\geqslant 1} $在$ x\ne k\pi $时不可能收敛到0.
	
	而当$ x=k\pi, k\in\Z $时, 可取$ x_n=x+\pi/2n $, 那么此时
	\[
	\abs{f_n(x)-f_n(x_n)}=\abs{\sin nx-\sin(nx+\pi)}=2\not\to 0,
	\]
	从而$ (\sin nx)_{n\geqslant 1} $在$ \R $上无处连续.\qed
	\end{Proof}
	\begin{Remark}
	实际上, 当$ x\ne k\pi, k\in Z $时, $ (\sin nx)_{n\geqslant 1} $不收敛到任何数. 用反证法, 设$ (\sin nx)_{n\geqslant 1} $收敛, 那么$ (\sin(n+1)x)_{n\geqslant 1} $也收敛. 由
	\[
	\sin(n+1)x=\sin nx\cos x+\cos nx\sin x
	\]
	可知$ (\cos nx)_{n\geqslant 1} $也收敛. 不妨设$ (\sin nx)_{n\geqslant 1} $与$ (\cos nx)_{n\geqslant 1} $分别收敛于$ a $和$ b $, 那么由$ \sin^2 nx+\cos^2 nx=1 $可知$ a, b $不能同时为零. 在
	\[
	\begin{cases}
	\sin(n+1)x=\sin nx\cos x+\cos nx\sin x\\
	\cos(n+1)x=\cos nx\cos x-\sin nx\sin x
	\end{cases}
	\]
	中令$ n\to \infty $可知
	\[
	\begin{cases}
	a=a\cos x+b\sin x\\
	b=b\cos x+a\sin x
	\end{cases}
	\]
	也即
	\[
	\begin{cases}
	(\cos x-1)a+\sin x\cdot b=0\\
	-\sin x\cdot a+(\cos x-1)b=0
	\end{cases}
	\]
	上面的方程组有非零解的充分必要条件是其系数行列式
	\[
	\det\begin{bmatrix}
	\cos x-1 & \sin x\\-\sin x & \cos x-1
	\end{bmatrix}=0,
	\]
	从而$ \cos x=1 $, 即$ x=2k\pi, k\in\Z $, 这与$ x\ne k\pi $矛盾, 因此$ (\sin nx)_{n\geqslant 1} $不收敛.
	\end{Remark}
	
	\textbf{习题5.5}\ [习题课]\ \ 考虑函数序列$ (f_n)_{n\geqslant 1} $, 这里$ f_n(t)=\sin\sqrt{t+4( n\pi )^2} $, 其中$ t\in[0,\infty) $.
	\begin{enumerate}[(1)]
	\item 证明$ (f_n)_{n\geqslant 1} $等度连续并且逐点收敛到0.
	\item 用$ C_b([0,\infty),\R) $表示$ [0,\infty) $上所有有界连续实函数构成的空间, 并赋予范数
	\[
	\norm{f}_\infty=\sup_{t\geqslant 0}\abs{f(t)},
	\]
	那么$ (f_n)_{n\geqslant 1} $在$ C_b([0,\infty),\R) $中是否相对紧?
	\end{enumerate}
	\begin{Proof}
	(1) 由Lagrange中值定理, 对任意的$ s, t>0 $, 有
	\[
	\abs{\frac{f(s)-f(t)}{s-t}}=\abs{f'(\xi)}=\abs{\frac{\cos\sqrt{\xi+4(n\pi)^2}}{2\sqrt{\xi+4(n\pi)^2}}}\leqslant\frac{1}{4\pi},
	\]
	即$ \abs{f_n(s)-f_n(t)}\leqslant\abs{s-t}/4\pi $, 从而$ (f_n)_{n\geqslant 1} $等度连续, 且
	\[
	\lim_{n\to\infty}f_n(x)=\lim_{n\to\infty}\sin\sqrt{t+4(n\pi)^2}=\lim_{n\to\infty}\sin(\sqrt{t+4(n\pi)^2}-2n\pi)=\lim_{n\to\infty}\sin\frac{t}{\sqrt{t+4(n\pi)^2}+2n\pi}=0,
	\]
	从而$ (f_n)_{n\geqslant 1} $逐点收敛到0.
	
	(2) 假设$ (f_n)_{n\geqslant 1} $相对紧, 则存在子列$ (f_{n_k})_{k\geqslant 1} $一致收敛到某$ f\in C_b([0,\infty),\R) $, 即$ \lim\limits_{k\to\infty}\norm{f_{n_k}-f}_\infty=0 $, 那么$ \forall t $都有$ \lim\limits_{k\to\infty}f_{n_k}(t)=f(t) $. 由(1)可知$ f=0 $, 又因为
	\[
	f_{n_k}(t)=\sin\frac{t}{\sqrt{t+4(n_{k}\pi)^2}+2n_{k}\pi}
	\]
	且
	\[
	\abs{\frac{t}{\sqrt{t+4(n_{k}\pi)^2}+2n_{k}\pi}}\leqslant 1,
	\]
	从而对$ \forall\varepsilon>0 $, 有$ \abs{f_{n_k}(t)}<\varepsilon $当且仅当$ t/(\sqrt{t+4(n_{k}\pi)^2}+2n_{k}\pi)<\arcsin\varepsilon $. 取$ \varepsilon=1/2 $, 则若记$ \theta=\pi/6 $
	\[
	\begin{aligned}
	\abs{f_{n_k}(t)}<\frac{1}{2} & \Longleftrightarrow \frac{t}{\sqrt{t+4(n_{k}\pi)^2}+2n_k\pi}<\theta \\
	& \Longleftrightarrow t-2n_k\pi\theta\leqslant\theta\sqrt{t+4(n_k\pi)^2} \\
	& \Longleftrightarrow (t-2n_k\pi\leqslant 0)\lor\left( \left( t-2n_k\pi\theta\geqslant 0 \right)\land\left( \left(t-2n_k\pi\theta\right)^2\leqslant\theta^2(t+4(n_k\pi)^2) \right) \right),
	\end{aligned}
	\]
	而后者等价于$ 2n_k\pi\theta\leqslant t\leqslant 4n_k\pi\theta+\theta^2 $, 故上式等价于$ t\leqslant 4n_k\pi\theta+\theta^2 $, 这意味着$ (f_{n_k})_{k\geqslant 1} $不可能一致收敛到0, 因此$ (f_n)_{n\geqslant 1} $在$ C_b([0,\infty),\R) $上不是相对紧的.\qed
	\end{Proof}
	\begin{Remark}
	本题说明了在定理\,\ref{thm:逐点->一致}\,和\,\ref{thm:Ascoli}\,中去掉$ K $的紧性后, 结论不再成立.
	\end{Remark}
	
	\textbf{习题5.10}\ [作业]\ \ 设$ (K,d) $是紧度量空间, 证明所有从$ K $到$ \R $的Lipschitz函数构成的集合在$ C(K,\R) $中稠密.
	\begin{Proof}
	设$ \CA $是$ K $到$ \R $的Lipschitz函数全体, 易证$ \CA $是一个线性空间, 且对任意$ f, g\in\CA $, 其中$ f $的常数为$ c_1 $, $ g $的常数为$ c_2 $, 由
	\[
	\begin{aligned}
	\abs{(f\cdot g)(x)-(f\cdot g)(y)}&\leqslant\abs{f(x)}\abs{g(x)-g(y)}+\abs{g(y)}\abs{f(x)-f(y)}\\
	&\leqslant(c_2\norm{f}_\infty+c_1\norm{g}_\infty)d_K(x,y),
	\end{aligned}
	\]
	且因$ K $是紧的而$ f, g $是连续的, 应有$ \norm{f}_\infty,\norm{g}_\infty $都是有限的, 从而$ f\cdot g\in\CA $, 也即$ \CA $是一个代数.
	
	取$ d_x(y)=d(x,y) $, 由三角不等式可知$ d_x $是常数为1的Lipschitz函数, 且$ x\ne y $时$ d_x(x)\ne d_x(y) $, 故$ \CA $是可分点的. 又有$ 1\in\CA $是显然的, 则由Stone-Weierstrass定理可知命题成立.\qed
	\end{Proof}

\textbf{习题5.15}\ [习题课]\ \ 设 $ f\in L_{2\pi}^{1} $, 其 \textbf{Fourier 级数}定义为
	\[
		f\sim \sum_{n=-\infty}^{\infty}\hat{f}(n)\exp(\imag nt).
	\]
	并定义相应的 $ n $\textbf{次部分和}为
	\[
		S_{n}(f)(t)=\sum_{k = -n}^{n}\hat{f}(k)\exp(\imag kt).
	\]
	在本习题中, 假设 $ f $ 属于分段的 $ C^{1} $ 函数类, 也就是说, $ f\in C_{2\pi} $ 且存在 $ [0, 2\pi] $ 上的划分 $ 0=a_{0}<a_{1}<\cdots a_{k}=2\pi $, 使得 $ f $ 在每一个子区间 $ (a_{j}, a_{j+1}) $ 内属于 $ C^{1} $ 函数类, 并且 $ f' $ 在点 $ a_{j} $ 的右极限和点 $ a_{j+1} $ 的左极限都存在, 证明 $ f $ 的Fourier级数依无穷范数收敛到 $ f $, 而且
	\[
		\norm{f-S_{N}(f)}_{\infty}\leqslant\sqrt{\frac{2}{N}}\norm{f'}_{2}, \quad N\geqslant1.
	\]
	注意这里的导数 $ f' $ 在 $ f $ 的不可导点取左右导数. 
	\begin{Proof}
		设 $ f $ 属于分段的 $ C^{1} $ 函数类, 可证 Newton-Leibniz公式到 $ f $ 仍然成立. 设 $ a<b\in[0, 2\pi] $, 设相应的 $ f $ 的划分 $ a=x_{0}<x_{x}<\cdots<x_{m}=b $, 则
		\[
			\int_{a}^{b}f'(t)\diff t=\sum_{i=1}^{k}\int_{x_{i-1}}^{x_{i}}f'(t)\diff t=\sum_{i=1}^{k}f(x_{i})-f(x_{i-1})=f(b)-f(a).
		\]
		类似地, 可证分部积分公式也成立, 由此, 对分段 $ C^{1} $ 的 $ u, v $, 因为 $ (uv)'=u'v+vu' $, 同取积分有
		\[
			\int_{a}^{b}u'v\diff t=\int_{a}^{b}(uv)'\diff t-\int_{a}^{b}uv'\diff t=uv\Big|_{a}^{b}-\int_{a}^{b}uv'\diff t
		\]
		求 $ f $ 的 Fourier 系数, 当 $ k\ne0 $ 时
		\[
			\begin{aligned}
				\widehat{f'}(k) & =\int_{0}^{2\pi}f'(\theta)\exp(-\imag k\theta)\frac{\diff \theta}{2\pi} \\
				& = \frac{f(\theta)\exp(-\imag k\theta)}{2\pi}\bigg|_{0}^{2\pi}+\imag k\int_{0}^{2\pi}f(\theta)\exp(-\imag k\theta)\frac{\diff \theta}{2\pi} = \imag k\hat{f}(\theta)
			\end{aligned}
		\]
		即 $ \widehat{f'}(k)=-\imag \hat{f}(k)/k $.

		再由题设: $ f'\in L_{2\pi}^{1} $, 由 Parseval 恒等式
		\[
			\int_{0}^{2\pi}\abs{f'(\theta)}^{2}\frac{\diff\theta}{2\pi}=\sum_{k\ne0}\abs{\widehat{f'}(k)}^{2}
		\]
		又
		\[
			\sum_{k\ne0}\abs{\hat{f}(k)}=\sum_{k\ne0}\abs{\frac{\widehat{f'}(k)}{k}}\leqslant\Big( \sum_{k\ne0}\frac{1}{k^{2}} \Big)^{1/2}\Big( \sum_{k\ne0}\abs{\widehat{f'}(k)}^{2} \Big)^{1/2}<\infty
		\]
		故 $ \sum\limits_{k=-\infty}^{\infty}\hat{f}(k)\exp(\imag kt) $ 一致收敛, 设极限为 $ g $, 往证 $ f=g $. 由 Parseval 恒等式
		\[
			\norm{S_{N}(f)-f}^{2}\to 0
		\]
		再由 Fatou引理
		\[
			0\leqslant\norm{g-f}_{2}\leqslant\lim_{N\to\infty}\norm{S_{N}(f)-f}_{2}=0
		\]
		故 $ \norm{g-f}_{2}=0 $, 即 $ g=f $. 至此得到
		\[
			\sum_{k=-\infty}^{\infty}\hat{f}(k)\exp(\imag kt)=f(t), \quad \forall t\in[0, 2\pi].
		\]
		设 $ J=\Z\setminus\{ -N, \dots, 0, \dots, N \} $, 则
		\[
			\begin{aligned}
				\abs{f(t)-S_{N}(f)(t)} & =\abs{\sum_{k\in J}(-\imag)\cdot\frac{\widehat{f'}(k)}{k}}\leqslant\sum_{k\in J}\abs{\frac{\widehat{f'}(k)}{k}} \\
				& \leqslant\Big( \sum_{k\in J}\frac{1}{k^{2}} \Big)^{1/2}\Big( \sum_{k\in J}\abs{\widehat{f'}(k)}^{2} \Big)^{1/2}\\
				& \leqslant \norm{f'}_{2}\Big( \sum_{k\geqslant N+1}\frac{2}{k^{2}} \Big)^{1/2}\leqslant\sqrt{\frac{2}{N}}\norm{f'}_{2}.
			\end{aligned}
		\]
		故
		\[
			\norm{f(t)-S_{N}(f)(t)}_{\infty}=\sup_{0\leqslant t\leqslant2\pi}\abs{f(t)-S_{N}(f)(t)}\leqslant\sqrt{\frac{2}{N}}\norm{f'}_{2}.
		\]
		命题得证.\qed
	\end{Proof}

	\textbf{习题5.17}\ [习题课]\ \ 对任意$ a\in\R $, 相应的平移变换$ \tau_a(f)(x):=f(x-a) $. 证明对任意$ f\in L_{2\pi}^p $和$ 0<p<\infty $, 都有
	\[
	\lim_{a\to 0}\norm{\tau_a(f)-f}_p=0.
	\]
	\begin{Proof}
	设$ f(x)=\exp(\imag nx) $, 其中$ n\in\N $, 那么
	\[
	\norm{\tau_a(f)-f}_p=\left(\int_0^{2\pi}\abs{\exp(\imag n(x-a))-\exp(\imag nx)}^p\diff x\right)^{1/p}=\abs{\exp(-\imag na)-1}\to 0\qquad (a\to 0)
	\]
	故$ \forall f\in T $, 都有$ \norm{\tau_a(f)-f}_p\to 0 $, 而$ \forall f\in L_{2\pi}^p $, 则
	\[
	\forall\varepsilon>0\,\exists g\in T\,\left(\norm{f-g}_p<\frac{\varepsilon}{3}\right)
	\]
	而
	\[
	\forall g\in T\,\exists\delta>0\,\left(\abs{a}<\delta\Rightarrow\norm{\tau_a(g)-g}_p<\frac{\varepsilon}{3}\right)
	\]
	于是
	\[
	\norm{\tau_a(f)-f}_p\leqslant\norm{\tau_a(f)-\tau_a(g)}_p+\norm{\tau_a(g)-g}_p+\norm{g-f}_p<\varepsilon.
	\]
	从而$ \forall f\in L_{2\pi}^p $, 都有$ \lim\limits_{a\to 0}\norm{\tau_a(f)-f}_p=0 $.\qed
	\end{Proof}